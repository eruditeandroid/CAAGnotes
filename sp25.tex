\documentclass{article}

\usepackage{notes}

\title{CAAG Seminar Notes, Spring 2025}
\author{Notes by John S.\ Nolan, speakers listed below}

\begin{document}

\maketitle

\tableofcontents

\pagebreak

\section{1/14a (Daniel Erman) -- Long Live the King (Conjecture)}

This is based on joint work with Ballard, Berkesch, Brown, Cranton Heller, Favero, Ganatra, Hanlon, and Huang.
We work throughout over an arbitrary algebraically closed field.

\subsection{Background on FSECs}

Write $\Dsf(\PP^n)$ for the bounded derived category of coherent sheaves on $\PP^n$.

\begin{thm}[Beilinson]
	The category $\Dsf(\PP^n)$ is generated by the \emph{Beilinson collection} $\Oc, \Oc(-1), \dots, \Oc(-n)$.
\end{thm}

Here ``generation'' is expressed in terms of taking direct sums, cones, and summands.

More generally, we can talk about full strong exceptional collections in a derived category.

\begin{dfn}
	A collection of objects $\Ec_1, \dots, \Ec_r$ in $\Dsf(X)$ is a \emph{full strong exceptional collection} (FSEC) for $\Dsf(X)$ if:
	\begin{enumerate}
		\item $\Ext^i(\Ec_j, \Ec_j) = \begin{cases}
				k & i = 0 \\
				0 & i \neq 0
			\end{cases}$
		\item For $k < j$ and all $i$, we have $\Ext^i(\Ec_k, \Ec_j) = 0$.
		\item The objects $\Ec_1, \dots, \Ec_r$ generate $\Dsf(X)$.
	\end{enumerate}
\end{dfn}

Beilinson's theorem states that the Beilinson collection is a FSEC for $\Dsf(\PP^n)$.
In particular, any coherent sheaf $\Fc$ on $\PP^n$ can be resolved by a complex $F$ where each $F_i = \oplus_{j=0}^n \Oc_{\PP^n}(-j)^{m_{ij}}$ (for some $m_{ij}$).
These resolutions and the corresponding invariants are quite useful for understanding the behavior of coherent sheaves on $\PP^n$.

A famous conjecture on the subject is the following:

\begin{conj}[King]
	Every smooth projective toric variety has a FSEC of line bundles.
\end{conj}

Unfortunately, this is false -- counterexamples were first found by Hille and Perling.
But similar statements are true (e.g.\ Kawamata constructed full exceptional collections of complexes).
The goal of this talk is to prove one analogue of King's conjecture.

\subsection{Cox rings of toric varieties}

One way to construct $\PP^n$ is as follows.
Let $S$ be the polynomial ring $k[x_0, \dots, x_n]$.
Then $\PP^n = (\Spec S \setminus V(x_0, \dots, x_n)) / \GG_m$.

More generally, given a nice (e.g.\ smooth) projective toric variety $X$, we let $S$ be the \emph{Cox ring} $\Cox(X) = \oplus_{d \in \Cl(X)} H^0(X, \Oc_X(d))$.
This is a polynomial ring with an interesting $\Cl(X)$-grading.
We can construct an irrelevant ideal $B \subset S$.
Then $X = (\Spec S \setminus V(B)) / G$, where $G$ is the commutative algebraic group with $\weight(G) = \Cl(X)$.

\begin{ex}
	Let $X$ be the Hirzebruch surface $\Hc_3 = \PP(\Oc_{\PP^1} \oplus \Oc_{\PP^1}(-3))$.
	Then $\Cl(X) = \ZZ^2$ and $S = k[x_0, x_1, x_2, x_3]$ where $\deg x_0 = (1, 0)$, $\deg x_1 = (-3, 1)$, $\deg x_2 = (1, 0)$, and $\deg x_3 = (0, 1)$.
	This grading corresponds to an action of $\GG_m^2$ on $\AA^4$ by
	\[
		(\lambda, \mu) \cdot (x_0, x_1, x_2, x_3) = (\lambda x_0, \lambda^{-3} \mu x_1, \lambda x_2, \mu x_3).
	\]
	The irrelevant ideal here is $B = (x_0, x_2) \cap (x_1, x_3)$.
\end{ex}

By changing the irrelevant ideal, we can obtain different quotients from the same $S$ with $\Cl(X)$-grading.
These are controlled by a ``GKZ fan'' in $\Cl(X)_\RR$ (with support given by the effective cone).
From the maximal cones $\Gamma_i$ of the GKZ fan, we obtain simplicial projective toric varieties $X_i = (\Spec S \setminus V(B_i)) / G$ with $\Nef(X_i) = \Gamma_i$.
Actually, we will usually consider the corresponding smooth toric Deligne-Mumford stacks.
These are smooth even when the simplicial toric varieties aren't, and we can identify $\Dsf(X_i)$ with the quotient of $\Dsf(S)$ by $B_i$-torsion modules.

\begin{ex}
	For the Hirzebruch surface $\Hc_3$, the GKZ fan has two maximal cones.
	One corresponds to $\Hc_3$, and the other corresponds to the weighted projective stack $\PP(1, 1, 3)$.
\end{ex}

\subsection{Cox categories}

The idea of this project is to glue the derived categories of toric stacks corresponding to maximal fans in the GKZ cone.
We define a category $\Dsf_{\Cox}$ as a ``union'' of the $\Dsf(X_i)$.
Then we can show that the natural analogue of the Beilinson collection is a FSEC of ``line bundles'' for $\Dsf_{\Cox}$.

To construct $\Dsf_{\Cox}$, let $\tilde{X}$ be a smooth DM toric stack with proper birational toric maps $\pi_i: \tilde{X} \to X_i$ (as $i$ ranges over the maximal GKZ cones $\Gamma_i$).
The functors $\Lbf \pi_i^*: \Dsf(X_i) \to \Dsf(\tilde{X})$ are fully faithful, and we define $\Dsf_{\Cox}$ as the smallest full triangulated subcategory of $\Dsf(\tilde{X})$ containing the images $\Lbf \pi_i^* \Dsf(X_i)$.
This construction turns out to be independent of the choice of $\tilde{X}$ -- in fact, $\Dsf_{\Cox}$ depends only on the Cox ring (with its grading).

The analogy here is that $\Dsf(X_i)$ is to $\Dsf_{\Cox}$ as affine patches are to projective varieties.
In particular, for neighboring $\Gamma_i$ and $\Gamma_j$, we obtain a birational map $X_i \dashrightarrow X_j$ and hence a Fourier-Mukai transform $\Phi_{ij}: \Dsf(X_i) \to \Dsf(X_j)$ (defined using the graph of the birational map).
These $\Phi_{ij}$ play the role of transition functions in constructing $\Dsf_{\Cox}$.

\subsection{The Bondal-Thomsen collection}

Let $\Theta_S = \bset{\sum_i a_i \deg(x_i)}{a_i \in (-1, 0]} \cap \Cl(S)$.
We call this the \emph{Bondal-Thomsen collection} of degrees in $\Cl(S)$.
We will also use $\Theta_S$ to denote the corresponding collection of line bundles.
This has many applications, most famously in Bondal's approach to mirror symmetry for toric varieties.

\begin{ex}
	If $S = k[x_0, \dots, x_n]$ with $\deg x_i = 1$ for all $i$, then $\Theta_S = \{ -n - 1, -n, \dots, 0 \}$.
\end{ex}

\begin{ex}
	If $S = \Hc_3$, then $\Theta_S$ consists of six line bundles which can be defined explicitly / understood via pictures.
\end{ex}

\begin{thm}
	For an appropriate ordering, the set $\Theta_S$ forms a FSEC for $\Dsf_{\Cox}$.
\end{thm}

This theorem comes from many different sources / motivations.
One is just attempting to understand how the algebra of the Cox ring captures the geometry of the toric varieties.
Another is symplectic geometry: the Cox category is mirror to a colimit of wrapped Fukaya categories obtained by taking the union of stop data.

The authors thought a bit about the relationship between this and Kawamata's theorem.
The speaker conjectured that there should be a simple proof of Kawamata's theorem using this machinery, but such a proof is not yet known.

\section{1/14b (Eric Larson) -- Normal Bundles of Rational Curves in Grassmannians}

Recall that any vector bundle on $\PP^1$ is isomorphic to a direct sum of line bundles $\Oc_{\PP^1}(a_i)$ (for unique integers $a_i$).

\begin{qn}
	Let $C \subset G(a, a + b)$ be a general rational curve.
	How does the normal bundle $N_C$ decompose as a direct sum of line bundles?
\end{qn}

The term ``general'' here requires irreducibility of the moduli space.
This holds for the cases we care about: globally generated vector bundles are non-special.

\subsection{A conjecture and some counterexamples}

We would hope that the following is true.

\begin{conj}[Na\"ive Conjecture]
	The normal bundle $N_C$ is \emph{balanced}: for all $i, j$, we have $\abs{a_i - a_j} \leq 1$.
\end{conj}

Unfortunately, this cannot be true in this generality.

\begin{ex}
	Consider a general curve of degree $2$ in $\PP^3$.
	Then $N_C \cong \Oc_{\PP^1}(2) \oplus \Oc_{\PP^1}(4)$.
\end{ex}

The problem here is due to degeneracy.
More generally, consider $G(a, a+b)$, and assume without loss of generality that $a \leq b$.
If $1 < d < a$, then
\[
	N_{C|G(a,a+b)} \cong N_{C|G(d,d+b)} \oplus Q|_C^{\oplus (a - d)},
\]
and this cannot be balanced because 
\[
	\mu(Q|_C) = \frac{d}{b} < 1 < \frac{(b+d) d - 2}{bd - 1} = \mu(N_{C|G(d,d+b)})
\]
Similarly, if $a < d < b$, then
\[
	N_{C|G(a,a+b)} \cong N_{C|G(a,a+d)} \oplus (S|C^\vee)^{\oplus (b - d)},
\]
and this cannot be balanced because
\[
	\mu(S|_C^\vee) = \frac{d}{a} < \frac{(a+d) d - 2}{ad - 1} = \mu(N_{C|G(a,a+d)}) - 1.
\]

Even when $d$ is large (so this degeneracy issue is avoided), other problems can arise.

\begin{ex}
	Consider rational curves in $\PP^b$.
	We have a short exact sequence
	\[
		\SES{N_{C|\PP^b}(1)}{\Oc_{\PP^1}^{b+1}}{P^1(\Oc_{\PP^1}(d))}.
	\]
	In characteristic 2, we can use this SES to show that $N_{C|\PP^b}(1)$ is a pullback under the Frobenius.
	We can then show that $N_{C|\PP^b}$ is balanced if and only if $d \equiv 1 \pmod{b+1}$.
\end{ex}

The above examples are the only ways for things to go wrong on $\PP^n$.
However, more phenomena can be found for general Grassmannians.

\begin{ex}
	Consider degree $7$ curves on $G(3, 6)$.
	Then $S|_C^\vee$ and $Q|_C$ are vector bundles of degree $7$ and rank $3$.
	Generically these look like $\Oc_{\PP^1}(2)^2 \oplus \Oc_{\PP^1}(3)$.
	Thus 
	\[
		T_{G(3,6)}|_C = \Oc_{\PP^1}(4)^4 \oplus \Oc_{\PP^1}(5)^4 \oplus \Oc_{\PP^1}(6).
	\]
	The na\"ive conjecture predicts that $N_C \cong \Oc_{\PP^1}(5)^8$, but one can see this is impossible by looking at the short exact sequence defining $N_C$.
\end{ex}

More generally, write $d = aq_1 + r_1 = bq_2 + r_2$ (with $0 \leq r_1 < a$ and $0 \leq r_2 < b$).
Then $S|C^\vee$ is generically $\Oc_{\PP^1}(q_1)^{a-r_1} \oplus \Oc_{\PP^1}(q_1 + 1)^{r_1}$, and $Q|_C$ is generically $\Oc_{\PP^1}(q_2)^{b-r_2} \oplus \Oc_{\PP^1}(q_2 + 1)^{r_2}$.
In this case, the above argument shows that the na\"ive conjecture must fail if $r_1 r_2 \neq 0$, $q_1 + q_2 \leq (a - r_1) (b - r_2)$, and $d \neq 1$.

\subsection{A corrected conjecture}

\begin{conj}[Coskun--Larson--Vogt]
	The normal bundle $N_C$ is balanced if and only if none of the following conditions hold:
	\begin{enumerate}
		\item (Degeneracy): $d < b$ (unless $d = 1$ or $d = a$).
		\item (Characteristic $2$): $\charop k = 2$ and $a = 1$ (unless $d \equiv 1 \pmod{b-1}$).
		\item (Tangent bundle): $r_1 r_2 \neq 0$ and $q_1 + q_2 \leq (a - r_1) (b - r_2)$ (unless $d = 1$).
	\end{enumerate}
\end{conj}

Essentially, we expect that the previously mentioned counterexamples are the only counterexamples.
Some evidence for this conjecture comes from computer experiments.
Another piece of evidence comes from the following.

\begin{dfn}
	Say that $\oplus_i \Oc_{\PP^1}(a_i)$ is \emph{$\ell$-balanced} if $|a_i - a_j| \leq \ell$ for all $i, j$.
\end{dfn}

\begin{thm}[Coskun--Larson--Vogt]
	The normal bundle $N_C$ is always $2$-balanced.
\end{thm}

Yet another piece of evidence is that the theorem holds for some low-rank Grassmannians in all degrees.

\begin{thm}[Cao]
	For $\charop k = 0$, the conjecture holds in $G(2, 4)$, $G(2, 5)$, and $G(2, 6)$.
\end{thm}

\subsection{Proof of $2$-balancedness}

The remainder of the talk will discuss the proof of the theorem of Coskun--Larson--Vogt.

Suppose we have a reducible curve $X \cup Y \subset Z$, where $X$ and $Y$ are rational curves intersecting in a point $p$.
How do we compute $N_{X \cup Y}|_X$?

If $E$ is a vector bundle on $X$ and $D$ is a Cartier divisor on $X$, then we can construct a \emph{negative modification}
\[
	E[D \xrightarrow{-} F] = \ker(E \to E|_D / F)
\]
and a \emph{positive modification}
\[
	E[D \xrightarrow{+} F] = E[D \xrightarrow{-} F](D).
\]

Assume that $a, b > 1$.
Suppose $V$ is a vector space of dimension $a + b$.
For $p \in \PP V$, define $\pi_p: G(a, V) \dashrightarrow G(a, V/p)$ as the natural rational map.
Let $N_{C \to p}$ be the image of $T \pi_p \to N_C$.
For $x \in C$, define $N_C[x \smile p] = N_C[x \xrightarrow{+} N_{C \to p}]$ (the $\smile$ is supposed to be an upside-down $\curvearrowright$).
Dually, for $h \in \PP V^\vee$, we can define $\pi^h$, $N^{c \to h}$, and $N_C[x \curvearrowright h]$.

Let $x \in C$, and let $L = \bset{\ell \in G(a, V)}{h \cap x \subset \ell \cap \ol{x + p}}$ (???).
The key proposition in the proof is that, if $N_C[x \smile p][x \curvearrowright h]$ is $\ell$-balanced, then so is the normal bundle of a deformation of $C \cup L$.

\begin{ex}
	To prove the theorem for a curve of degree $6$ in $G(2, 4)$, we start with a curve $C'$ of degree $2$ and four $L_i$.
	We then apply the result four times and specialize all $p_i$ to a common point $p$.
	One can then do an explicit computation using these and a SES to understand $N_C$.
\end{ex}

In most cases, this approach actually proves balancedness.

\section{1/21a (Reed Jacobs) -- Borel-Weil(-Bott) Theory}

Recall that a Lie group is a manifold with group structure such that the structure maps are smooth.
Given a Lie group $G$, we can obtain a Lie algebra $\gfr = \Lie(G) = T_1 G$ together with a Lie bracket $[-,-]: \gfr \times \gfr \to \gfr$.
In most examples of interest, $\gfr$ is some space of matrices with $[-,-]$ given by commutator.

\begin{ex}
	If $G = \GL_n$, then $\gfr$ is the space of all $n \times n$ matrices.
\end{ex}

\begin{ex}
	If $G = \SL_n$, then $\gfr$ is the space of all traceless $n \times n$ matrices.
\end{ex}

\subsection{Tori and Borels}

\begin{dfn}
	Let $G$ be a Lie group.
	A \emph{torus} $T \subset G$ is a compact connected abelian subgroup.
\end{dfn}

\begin{ex}
	For $G = \SL_2$, we can take
	\[
		T = \bset{\begin{bmatrix} z & 0 \\ 0 & \ol{z}\end{bmatrix}}{z \in \CC, z \ol{z} = 1}.
	\]
\end{ex}

The Lie algebra $\tfr = \Lie(T)$ is always abelian (i.e.\ it has the trivial Lie algebra structure).
We have a natural covering map $\pi: \tfr \to T$, and because $T$ is compact, $\ker(\pi)$ must be discrete.
Thus $T$ may be identified with a quotient $\RR^r / \ZZ^r$.

The \emph{rank} of a torus $T$ is just $\dim T$.
This is bounded above by $\dim G$, so we may speak of \emph{maximal tori}, i.e.\ tori with maximal rank.

\begin{thm}
	All maximal tori are conjugate.
\end{thm}

\begin{proof}
	This follows from the bijection between tori of $G$ and \emph{Cartan subalgebras} in $\gfr$.
	(Over $\CC$, a \emph{Cartan subalgebra} is a maximal subalgebra $\hfr \subset \gfr$ such that, for every $x \in \hfr$, the element $x$ acts semisimply on $\gfr$.)
\end{proof}

We may also consider solvable subgroups instead of abelian subgroups.

\begin{dfn}
	A \emph{Borel subgroup} of $G$ is a maximal connected solvable subgroup.
\end{dfn}

\begin{thm}
	Any two Borels are conjugate.
\end{thm}

\subsection{The case of $\SL_2$}

Let $G = \SL_2(\CC)$, and let $X = \CC\PP^1$.

One way to obtain all irreps of $\SL_2(\CC)$ is the following.
For $n \geq 0$, let $A_n$ be the vector space of degree $n$ homogeneous polynomials in two variables (so $\dim A_n = n + 1$).
Let $V$ be the standard representation of $\SL_2(\CC)$, so that $\Sym (V^\vee) = \CC[x, y]$.
Then $\Sym (V^\vee) = \oplus_{n \geq 0} A_n$, and we obtain actions $\SL_2(\CC) \curvearrowright A_n$ for all $n$.
These are all of the irreps.

Here's an interesting coincidence: on $X = \CC\PP^1$, for all $n \in \ZZ$, we have a line bundle $\Oc(n)$.
If $n \geq 0$, then we have $\Gamma(X, \Oc(n)) = A_n$.
Thus the irreps of $\SL_2(\CC)$ are the spaces of global sections of positive line bundles on $\CC\PP^1$.
What's going on here?

Let's fix a Borel $B \subset \SL_2(\CC)$, e.g.\
\[
	B = \bset{\begin{bmatrix} x & y \\ 0 & x\inv \end{bmatrix}}{x \in \CC^\times, y \in \CC}.
\]
The quotient $G / B$ may be identified with $\CC\PP^1$, where the coset of 
\[
	\begin{bmatrix} a & b \\ c & d \end{bmatrix} \in \SL_2(\CC)
\]
corresponds to $[a : c] \in \CC\PP^1$.
This at least indicates that there should be some interesting mathematics underlying the above coincidence.

\subsection{Generalizing the above coincidence}

More generally, suppose $G$ is a complex semisimple Lie group, $T \subset G$ is a maximal algebraic torus\footnote{We have switched to the algebraic setting here.}, and $B \supset T$ is a Borel.
Then:

\begin{thm}
	For any complex semisimple Lie group $G$ and any Borel $B$, the quotient $G / B$ is a projective variety (and is independent of $B$ up to isomorphism).
\end{thm}

If $U$ is the unipotent radical of $G$, then $B / U = T$.
Letting $\lambda: T \to \CC^\times$ be a holomorphic character, we can use the preceding identification to pull back $\lambda$ to a character $\tilde{\lambda}: B \to \CC^\times$.
Thus a character of $T$ gives rise to a one-dimensional representation of $B$.
Viewing $G \to G / B$ as a principal $B$-bundle, taking the associated bundle corresponding to the representation from the character $\lambda$ gives a line bundle $\Lc_\lambda$ on $G / B$.
In general, we want to study $\Gamma(G / B, \Lc_\lambda)$.

\begin{thm}[Borel-Weil]
	Let $G$ be a complex semisimple Lie group, $T \subset G$ a maximal torus, and $B \supset T$ a Borel.
	Let $\lambda$ be a dominant integral weight of $T$.
	Then $\Gamma(G / B, \Lc_\lambda)$ is the unique irrep of $G$ with highest weight $\lambda$.
\end{thm}

\subsection{1/21b (Ben Church) -- Curves on Complete Intersections and Measures of Irrationality}

This is based on joint work with Nathan Chen and Junyan Zhao.

Let $X \subset \PP^{n+r}$ be a general complete intersection of type $(d_1, \dots, d_r)$ (i.e.\ $X$ is cut out by polynomials of degree $d_1, \dots, d_r$).
We want to understand the curves on $X$.
More precisely:

\begin{qn}
	For a curve $C \subset X$, what constraints exist for the degree / genus / gonality?
\end{qn}

\subsection{Linear slice curves}

One family of curves on $X$ consists of \emph{linear slice curves} $C_\Lambda = X \cap \Lambda$, where $\Lambda \subset \PP^{n+d}$ is a linear subspace of dimension $r - 1$.
For general $X$, one expects that these are the ``simplest'' curves on $X$.

\begin{thm}[Chen-C-Zhao, 2024]
	Let $X$ be a general complete intersection of type $(d_1, \dots, d_r)$.
	There exists $N(n, r)$ such that, if $d_i \geq N$ for all $i$, then $\deg C \geq d_1 \dots d_r$.
	Moreover, if $d_i \geq 2n + 1$ for all $i$, then $\deg C \geq (d_1 - 2n + 1) \dots (d_r - 2n + 1)$.
\end{thm}

Some results on this subject were known beforehand.
The Noether-Lefschetz theorem implies that, if $X$ is a surface with all $d_i \geq 4$, then $d_1 \dots d_r | \deg C$.
A theorem of Wu from 1990 states that, if $X$ is a 3-fold hypersurface with $d \geq 6$ and $\deg C < 2d - 1$, then $C$ is a linear slice curve.
Voisin proved that $X$ always has a non-complete intersection curve whenever $n \geq 3$.

\subsection{Measures of irrationality}

How can we quantify ``how irrational'' a variety $X$ is?
There are many possible approaches which generalize the gonality of a curve.

\begin{dfn}
	The \emph{degree of irrationality} $\irr(X)$ is the minimum $\delta$ such that there exists a $\delta$-to-$1$ map $X \dashrightarrow \PP^{\dim X}$.
\end{dfn}

\begin{dfn}
	The \emph{covering gonality} $\covgon(X)$ is the smallest $c$ such that there always exists a curve $C$ of gonality $c$ through a general point of $X$.
\end{dfn}

Note that $\irr(X) = 1$ if and only if $X$ is rational, and $\covgon(X) = 1$ if and only if $X$ is uniruled.
In general we have $\irr(X) \geq \covgon(X)$, so we will focus on providing bounds for $\covgon(X)$.

\begin{thm}[BDELU, 2017]
	Let $X_d \subset \PP^{n+1}$ be a very general smooth hypersurface of degree $d \geq n+2$.
	Then $\covgon(X_d) \geq d - n$ and $\irr(X_d) = d - 1$ (?).
\end{thm}

More generally, the methods of BDELU show that $\covgon(X_{d_1,\dots,d_r}) \geq d_1 + \dots + d_r - n$.
This relates to the positivity of the canonical bundle: $K_{X_{d_1,\dots,d_r}} = \Oc(d_1 + \dots + d_r - n - r - 1)$.
BDELU also ask whether there exists a multiplicative bound, i.e.\ one of the form $\covgon(X) \geq K d_1 \dots d_r$.

\begin{thm}[Chen-C-Zhao, 2024]
	Let $X \subset \PP^{n+r}$ be a general complete intersection of type $(d_1, \dots, d_r)$, and suppose $d_i \geq n$.
	Then $\covgon(X) \geq (d_1 - 2n \sqrt{d_1}) (d_2 - n + 1) \dots (d_r - n + 1)$.
\end{thm}

\subsection{How to prove the first theorem}

For simplicity, let's consider a hypersurface $X_d$.
We'd like to degenerate $X_d$ to $X_a \cup_Z X_b$, where $a + b = d$ and $Z$ is a complete intersection of type $(a, b)$.
(If $X_d = V(f)$, this can be accomplished by considering a family $V(Ft - HG)$ parametrized by $t$.)
Given such a degeneration, we can hope that a curve $C$ on $X_d$ breaks into a curve $C_1$ on $X_a$ and $C_2$ on $X_b$.
Then we would obtain
\[
	\deg C = \deg C_1 + \deg C_2 \geq \min \deg(X_a) + \min \deg (X_b),
\]
and we could use an inductive approach to prove the result.

Unfortunately, this degeneration often fails to break our curve $C$.

\begin{ex}
	Consider a degeneration of a quadric into a union of hyperplanes.
	There are two families of lines on the quadric, but general curves end up lying in only one of the hyperplanes.
	The speaker drew a beautiful picture illustrating the fact that all curves actually pass through one of two special points on the intersection of the hyperplanes.
	Not every curve on the union of hyperplanes can be obtained as a limit of curves from the quadric.
\end{ex}

Let $\Xc$ be the total space of the degeneration, and let $C'$ be a curve on $X_0 = X_1 \cup_Z X_2$.
If $p \in Z \cap C'$, then one of the following must hold:
\begin{enumerate}
	\item $p$ is a node of $C'$ and meets components of both $X_1$ and $X_2$.
		In this case, we get $\deg C \geq \min \deg(X_1) + \min \deg(X_2)$.
	\item $p$ lies in a curve entirely contained in $Z$.
		In this case, we get $\deg C \geq \min \deg(Z)$.
	\item $p$ lies in the singular locus of $\Xc$.
		It takes some cleverness to avoid this (this is the worst possible case).
\end{enumerate}

We combine these with an inductive argument to establish the theorem

\subsection{Bounding the gonality of curves}

If $X$ is smooth and projective, and sections of $\omega_X$ separate $\gamma$ points, then $H^0(\omega_X) \to k(x_1) + \dots + k(x_\gamma)$ is surjective.
It follows that $\covgon(X) \geq \gamma + 1$ (otherwise we could use the gonal map to construct a nontrivial holomorphic section of $\Oc_{\PP^{\dim X}}$).
Thus it suffices to show that $\omega_X$ separates many points.

More generally, one considers $X \subset Y$ of types $(d_1, \dots, d_r)$ and $(d_2, \dots, d_r)$ respectively.
The adjunction formula gives $K_X = (K_Y + X)|_X$.
To show that $\omega_X$ separates points $x_1, \dots, x_\gamma$, it suffices to show that $H^1(\Oc(K_Y + X) \otimes I_{x_1, \dots, x_\gamma}) = 0$.
This follows from a clever application of the Nadel vanishing theorem.
As part of this, the authors show:

\begin{thm}[Chen-C-Zhao, 2024]
	Let $(X, H)$ be a polarized smooth variety, and let $\alpha > 0$ be such that $\deg_H C \geq \alpha$ for all $C \subset X$.
	Then there exists $d_0$ (depending on $\dim X$ and $\alpha$) such that $|K_X + dH|$ separates $\alpha (d - 2 \dim X \sqrt{d})$ points.
\end{thm}

\section{1/28a (Mahrud Sayrafi) -- Oda's Problem for Toric Projective Bundles}

\subsection{Background on Oda's problem}

Given convex lattice polytopes $P$ and $Q$ in $\RR^d$, we always have $(P \cap \ZZ^d) + (Q \cap \ZZ^d) \subset (P + Q) \cap \ZZ^d$, where $+$ is Minkowski sum.
In general, the inclusion is strict.

Oda asked some relevant combinatorial questions:

\begin{qn}[Oda 1]
	When does equality hold?
\end{qn}

We know that equality holds if $d = 1$.
Other partial results are also known.

We can also try to limit the problem to the setting when $Q$ is obtained by independently parallel transporting the facets of $P$.
(There are pictures but I won't draw them here just yet.)

\begin{qn}[Oda 2]
	Do we have
	\[
		(P \cap \ZZ^d) + (k \cap \ZZ^d) \subset (k+1) P \cap \ZZ^d
	\]
	for $k > 0$?
\end{qn}

\subsection{Toric geometry}

Recall that convex lattice polytopes $P$ give rise to pairs $(X, D)$, where $X$ is a projective toric variety and $D$ is an ample divisor on $X$.
To get $X$ from $P$, we take the \emph{normal fan} $\Sigma_P$ of $P$ (with rays given by the normals to facets of $P$ and higher cones determined by adjacency relations between facets).
Other polytopes correspond to ample / nef / effective divisors on $X$, and the corresponding line bundles can be understood by counting lattice points.

We can translate the limited version of Oda's problem into the language of toric varieties as follows.

\begin{qn}[Oda 3]
	Let $X$ be a smooth projective toric variety, $D$ ample, and $D'$ nef.
	When is $H^0(X, \Oc(D)) \otimes_k H^0(X, \Oc(D')) \to H^0(X, \Oc(D + D'))$ surjective?
\end{qn}

This relates to some classical questions in algebraic geometry.
Recall:

\begin{dfn}
	A projectively embedded variety $X \hookrightarrow \PP^n$ is \emph{projectively normal} if $X$ is normal and the affine cone of $X$ is normal.
	We also say that an abstract projective variety $X$ is projectively normal if $X$ is projectively normal when embedded via any complete linear series.
\end{dfn}

If (Oda 3) above holds for \emph{all} ample $D$ and nef $D'$ on a projective toric variety $X$, then $X$ is projectively normal.

\begin{ex}
	The embedding $\PP^1 \subset \PP^1$ is projectively normal.
\end{ex}

\begin{ex}
	The twisted cubic $\PP^1 \subset \PP^3$ is projectively normal.
\end{ex}

\begin{ex}
	The embedding $\PP^1 \hookrightarrow \PP^3$ via
	\[
		[s : t] \mapsto [s^4 : s^3 t : s^2 t : s^5]
	\]
	is not projectively normal.
\end{ex}

\begin{conj}[Oda]
	Smooth projective toric varieties are projectively normal (when embedded via complete linear series).
\end{conj}

There are non-smooth examples for which the conjecture fails.

\subsection{Results}

\begin{thm}[Sayrafi]
	Let $X = \PP(\Ec_m) \to \PP(\Ec_{m-1}) \to \dots \to \PP(\Ec_0) = \PP^n$ with $\Ec_i$ sums of line bundles and $\rk \Ec_i = 2$ for all $i > 0$, then $X$ is 
\end{thm}

The $X$ described above is an iterated $\PP^1$-bundle over $\PP^n$.
The speaker hopes to remove the $\rk \Ec_i = 2$ hypothesis.

\begin{rmk}
	Let $X = \PP^{n_1} \times \dots \times \PP^{n_r}$.
	Then we can define a \emph{Cox ring}
	\[
		S = \oplus_{\dbf \in \ZZ^r} H^0(X, \Oc(\dbf)) = \CC[x_1, \dots, x_s]
	\]
	which is naturally graded by $\ZZ^r = \Pic X$.
	Let $M$ be a finitely generated $\Pic X$-graded $S$-module.
	Then:
	\begin{itemize}
		\item If $\dbf$ lies in the ``(multigraded) regularity of $M$'' (see Maclagan-Smith 2004), then the minimal free resolution of $M$ is ``quasilinear.''
			(For $X = \PP^n$, the minimal free resolution is linear.)
		\item In this case, the minimal free resolution of $M_{\geq \dbf}$ can be obtained from a shifted Fourier-Mukai transform of the $S$-module $\tilde{M}(d)$, where the Fourier-Mukai kernel is Beilinson's resolution of the diagonal.
	\end{itemize}
\end{rmk}

By using a certain resolution of the diagonal on the ``Bott towers'' $X$ above, Oda's conjecture reduces to showing the relationship between the minimal free resolution and Fourier-Mukai transform sketched above.

\section{1/28 (Eugene Gorsky) -- Compactified Jacobians and Affine Springer Fibers}

\subsection{Plane curve singularities}

Let $C$ be a (reduced, irreducible) plane curve with a singularity at $0$.
We may describe $C$ as $V(f) \subset \CC^2$, so the singularity is described by $f \in \CC[[x, y]]$.
We may also parametrize $C$ via $x(t), y(t) \in \CC[[t]]$.

Let $\Oc_{C,0}$ be the completed local ring $\CC[[x,y]] / (f) \cong \CC[[x(t), y(t)]]$.
The integral closure $\ol{\Oc_{C,0}}$ is $\CC[[t]]$.

\begin{dfn}
	The \emph{$\delta$-invariant} of the singularity is $\delta(C) = \dim \ol{\Oc_{C,0}} / \Oc_{C,0}$.
\end{dfn}

\begin{ex}
	For $f = x^m - y^n$ with $\gcd(m, n) = 1$, we have a parametrization $x = t^n, y = t^m$.
	Here $\delta(C) = (m - 1) (n - 1) / 2$.
\end{ex}

\begin{ex}
	It is a good exercise to find an equation corresponding to the parametrization $x = t^4$, $y = t^6 + t^7$.
	Here $\delta(C) = 8$.
\end{ex}

\subsection{Compactified Jacobians}

\begin{dfn}
	Let $\ol{JC}$ be the moduli space of rank-one torsionfree sheaves on $C$ with framing.
	Algebraically, $\ol{JC}$ parametrizes $\Oc_{C,0}$-submodules $M \subset \CC((t))$ (remembering the embedding!), normalized so that
	\[
		\dim M / t^N \CC[[t]] = \dim \Oc_{C,0} / t^N \CC[[t]]
	\]
	for $N \gg 0$.
\end{dfn}

If we didn't include the normalization data, we would get the ``compactified Picard group'' of $C$.

\begin{ex}
	For the cuspidal cubic $C = V(x^2 - y^3)$, we look for $M \subset \CC((t))$ such that $t^2 M \subset M$ and $t^3 M \subset M$ (satisfying the normalization condition above).
	Such modules are one of
	\[
		M = (t^k + \lambda t^{k+1}, t^{k+2}, t^{k+3}, \dots)
	\]
	for varying $\lambda$, or
	\[
		M = (t^k, t^{k+1}, t^{k+2}, \dots).
	\]
	The normalization picks a particular value of $k$.
	Thus $\ol{JC} = \CC \cup \{ \pt \} = \PP^1$.
\end{ex}

\begin{thm}[Altman, Iarrobino, Kleiman]
	If $C$ is reduced and irreducible, then $\ol{JC}$ is an irreducible projective variety of dimension $\delta(C)$.
	This contains the \emph{Jacobian} $JC$, which parametrizes free $\Oc_{C,0}$-modules (and is isomorphic to $\AA^{\delta(C)}$).
\end{thm}

\subsection{Homology of compactified Jacobians}

We'd like to understand the homology $H^*(\ol{JC})$.

\begin{conj}[Kottwitz]
	For any reduced, irreducible $C$, the homology $H^*(\ol{JC})$ is supported in even degrees.
\end{conj}

The speaker is somewhat skeptical of the stronger conjecture that $\ol{JC}$ can be stratified by affine spaces.
However, some others believe the conjecture.

\begin{conj}[Oblomkov, Rasmussen, Shende]
	Roughly: $H^*(\ol{JC})$ is isomorphic to the Khovanov-Rozansky homology $HHH(C \cap S^3)$ (where $C \cap S^3$, for small $S^3$, is the link of $C$).
	In particular, $H^*(\ol{JC})$ is determined \emph{only} by the topology of the link.
\end{conj}

\begin{thm}[Kvinen-Tsai 2022]
	Let $C$ be reduced and irreducible.
	The point count of $\ol{JC}$ over $\FF_q$ is a polynomial in $q$ with nonnegative integer coefficients.
	Moreover, these coefficients can be described combinatorially.
\end{thm}

The proof relies on serious number theory (orbital integrals, etc.).

\subsection{Affine pavings for generic $C$}

For nice enough $C$, we can prove the conjectures.

\begin{dfn}
	Say that $C$ is \emph{generic} if it has a parametrization $x = t^n$, $y = t^m + \lambda t^{m+1} + \dots$ and either $\gcd(m, n) = 1$ or $\lambda \neq 0$.
\end{dfn}

The following was proved by Pionkowski for $\gcd(m, n) = 1$.

\begin{thm}[Gorsky-Mazin-Oblomkov 2022]
	For generic $C$, $\ol{JC}$ has an affine paving with cells corresponding to lattice paths below the diagonal of an $m \times n$ rectangle.
	There is a combinatorial formula for the dimension of the cells.
\end{thm}

\begin{ex}
	For the singularity with parametrization $x = t^4$, $y = t^6 + t^7$, we can count that there are 23 lattice paths in a $4 \times 6$ rectangle.
	In particular, $\chi(\ol{JC}) = 23$.
\end{ex}

\begin{thm}[Caprau-Gonzalez-Hogancamp-Mazin]
	For generic curves, $H^*(\ol{JC}) = HHH(C \cap S^3)$.
\end{thm}

\begin{proof}
	Compute both sides and show that they are abstractly isomorphic.
\end{proof}

To construct the affine paving, consider the standard valuation on $\CC((tt))$: $v(a_k t^k + \dots) = k$ whenever $a_k \neq 0$.
For $M \in \ol{JC}$, we can use this to define the image subset $v(M) \subset \ZZ$.
Let $\Sigma_\Delta = \bset{M}{v(M) = \Delta}$ for a fixed subset $\Delta \subset \ZZ$.

\begin{thm}
	For all $\Delta$, either $\Sigma_\Delta$ is empty or it is an affine space.
\end{thm}

\begin{ex}
	Consider the singularity with parametrization $x = t^4$, $y = t^6 + t^7$.
	Let $\Delta = 2 \NN \subset \ZZ$.
	Can we find an $M$ such that $v(M) = \Delta$?
	Letting $f_0 = 1 + a t + \dots$ and $f_2 = t^2 + b t^3 + \dots$, we obtain
	\begin{align*}
		y f_0 - x f_2 &= (a + 1 - b) t^7 + \dots \\
		x^2 f_0 - y f_2 &= (a - b - 1) t^9 + \dots \\
	\end{align*}
	But $7, 9 \not\in \Delta$, so we must have $a + 1 - b = 0$ and $a - b - 1 = 0$.
	However, these systems do not have a simultaneous solution!
	Thus $\Sigma_\Delta = \varnothing$.
\end{ex}

A similar (but harder) method is used to prove the theorem in general.

\section{2/18 (Nathaniel Gallup) -- Semigroup-Graded Stillman's Conjecture}

This is based on joint work with John Cobb and John  Spoerl.

\subsection{Background}

Recall that the projective dimension of a module $M$, denoted $\pdim(M)$, is the minimum $\ell$ such that there exists a length $\ell$ projective resolution of $M$.

\begin{thm}[Hilbert Syzygy Theorem]
	Let $M$ be a finitely generated $k[x_1, \dots, x_n]$-module.
	Then $\pdim(M) \leq n$.
\end{thm}

There's an interesting related question: what if we allow the number of variables ($n$) to vary but only consider $M$ arising as homogeneous ideals of bounded degree?

\begin{conj}
	For all $r, d \in \NN$, there exists $B(r, d)$ such that, for all $f_1, \dots, f_r \in k[x_1, \dots, x_n]$ with $\deg f_i \leq d$, we have $\pdim(f_1, \dots, f_r) \leq B(r, d)$.
	This bound is independent of $n$.
\end{conj}

This has now been proved four times:
\begin{itemize}
	\item First in Ananyan-Hochster (2016)
	\item Twice in Erman-Sam-Snowden (2019)
	\item Once in Draisma-Lason-Leykin (2019)
\end{itemize}

Motivated by toric geometry, we would like to understand whether it is possible to prove a semigroup-graded version of Stillman's conjecture.

\subsection{Semigroup-graded commutative algebra}

Let $\Lambda$ be an abelian monoid.
A \emph{$\Lambda$}-grading of $S = k[x_1, x_2, \dots]$ is a decomposition $S = \oplus_{g \in \Lambda} S_g$ such that each $x_i$ lies in some $S_{g_i}$ and $S_g \cdot S_h \subset S_{g + h}$.
We say that $S$ is \emph{connected} if $S_0 = k$.

Some properties of monoids will be relevant to us:
\begin{itemize}
	\item $\Lambda$ is \emph{pointed} if $g + h = 0$ implies $g = h = 0$.
		This produces a divisibility order $\leq_\Lambda$ on $\Lambda$: $g \leq_\Lambda g'$ iff $g + h = g'$ for some $h \in \Lambda$.
	\item $\Lambda$ has \emph{bounded factorization} (BF) if, for all $g \in \Lambda$, there exists $N \in \NN$ such that $g = g_1 + \dots + g_s$ (with all $g_i \neq 0$) implies $s \leq N$.
\end{itemize}

\begin{ex}
	The monoid $\ZZ^n_{\geq 0}$ has bounded factorization.
\end{ex}

\begin{ex}
	The monoid $0 \sqcup \bset{(i,j)}{j \geq i} \subset \ZZ^2$ has bounded factorization (but is not downward-finite with respect to $\leq_\Lambda$).
\end{ex}

\begin{ex}
	The monoid $\QQ_{\geq 0}$ does not have bounded factorization.
\end{ex}

We assume $S$ is connected and $\Lambda$ is pointed throughout.

Let us say that $\Lambda$ has a \emph{Stillman bound} if, for all $r \in \NN$ and all $g \in \Lambda$, there exists a bound $B(r, g) \in \NN$ such that, for all $\Lambda$-graded polynomial rings $S = k[x_1, \dots, x_n]$ and all $f_1, \dots, f_r \in S$ with $\deg_\Lambda f_i \leq g$, we have $\pdim(f_1, \dots, f_r) \leq B(r, g)$.

\begin{thm}[Cobb-G-Spoerl 2024]
	$\Lambda$ has a Stillman bound if and only if $\Lambda$ has bounded factorization.
\end{thm}

This can be proved using the ultraproduct techniques of Erman-Sam-Snowden, but there's also a quick proof that reduces this to the usual Stillman conjecture.

\subsection{Proof of theorem}

($\Leftarrow$).
Let $S_n = k[x_1, \dots, x_n, y_1, \dots, y_n]$, and let $I_n = (f_1, f_2, f_3)$ where $f_1 = \prod_i x_i$, $f_2 = \prod_i y_i$, and $f_3 = \sum_i \prod_{j \neq i} x_j y_i$.
A result of McCullough-Seceleanu (2012) tells us that $\pdim(I_n) = n + 2$.
If $\Lambda$ does not have bounded factorization, then there exists $h \in \Lambda$ such that, for all $n \in \NN$, we can write $h = h_{n,1} + \dots + h_{n,n}$ for some nonzero $h_{n,k}$'s.
Let $r = 3$ and let $g = 2h$.
Given any $B \in \NN$, grade $S_B$ by $\deg_\Lambda x_i = \deg_\Lambda y_i = h_{B,i}$
Then $\deg_\Lambda f_1 = \deg_\Lambda f_2 = h \leq g$ and $\deg_\Lambda f_3 \leq 2h = g$.
However, $\pdim(I_B) = B + 2 > B$, so there is no Stillman bound.

($\Rightarrow$).
Suppose $\Lambda$ has bounded factorization.
For all $r \in \NN$ and all $g \in \Lambda$, let $d$ bound the size of all factorizations of $g$.
The condition $\deg_\Lambda(x_1^{e_1} \dots x_m^{e_m}) \leq g$ is equivalent to the condition that $e_1 \deg_\Lambda x_1 + \dots + e_m \deg_\Lambda x_m + h = g$ for some $h \in \Lambda$.
This implies $e_1 + \dots e_m \leq d$, i.e.\ the degree of $x_1^{e_1} \dots x_m^{e_m}$ with respect to the standard grading is $\leq d$.
By Stillman's conjecture, there exists $B(r, d)$ bounding $\pdim(f_1, \dots, f_r)$ whenever $\deg f_i \leq d$ for the standard grading, and in particular when $\deg_\Lambda f_i \leq d$.

\subsection{Ultraproducts}

\begin{lem}[Ananyan-Hochster 2016]
	$\Lambda$ has a Stillman bound if and only if, for all $r \in \NN$ and $g \in \Lambda$, there exists $B(r, g) \in \NN$ such that, for all $\Lambda$-graded $S = k[x_1, \dots, x_n]$ and all $f_1, \dots, f_r$ with $\deg_\Lambda f_i \leq g$, if the collective strength of the $f_i$ is $\geq B(r, g)$, then $f_1, \dots, f_r$ is a regular sequence.
\end{lem}

We won't define collective strength here.
However, note that infinite collective strength implies the $f_i$ are $k$-independent modulo $(S_+)^2$.

This can be proved loosely as follows: assuming it's not true, one obtains a family of counterexamples for all $B \in \NN$.
These live in the ultraproduct $\SS^\NN$.

\begin{thm}[Erman-Sam-Snowden 2019]
	$\SS^\NN$ is a polynomial ring.
\end{thm}

We end up with a collection in $\SS^\NN$ with infinite collective strength.
This implies that this collection is part of a polynomial basis, hence forms a regular sequence.
One can then argue that some of the original ``counterexamples'' must have been regular sequences.

Working in the $\Lambda$-graded context, one can prove similar results.

\begin{thm}[Cobb-G-Spoerl 2025]
	$\SS^\Lambda$ is a polynomial ring if and only if $\Lambda$ has bounded factorization.
\end{thm}

\section{2/25 (Noah Olander) -- On Weakly \'Etale Morphisms}

This is based on joint work with de Jong.

\subsection{Background}

Recall that a morphism of schemes is \emph{\'etale} if and only if any of the following equivalent conditions hold:
\begin{enumerate}
	\item $f$ is flat, locally of finite presentation, and unramified (i.e.\ the diagonal $\Delta_X$ is an open immersion)
	\item $f$ is locally of finite presentation and formally \'etale.
\end{enumerate}

\begin{rmk}
	The latter characterization is \emph{functorial}, i.e.\ straightforward to check for functors of points.
\end{rmk}

\'Etale morphisms can be used to define $\ell$-adic cohomology.
More precisely, if $X$ is a scheme, we let $X_\et$ be the category of schemes \'etale over $X$, with coverings given by surjective \'etale morphisms $V \to U$.
If $\ell$ is an integer invertible in $\Oc(X)$, we let $H^i(X, \ZZ_\ell) = \lim_n H^i(X_\et, \ul{\ZZ / \ell^n})$.

\'Etale cohomology looks like it should be the ``cohomology of $\ul{\ZZ_\ell}$,'' but the ``sheaf'' $\ul{\ZZ_\ell}$ does not make sense.
Bhatt and Scholze fixed this by defining a ``pro-\'etale'' site on which $\ul{\ZZ_\ell}$ is a sheaf.
This relies on the following notion.

\begin{dfn}
	A morphism $f: X \to Y$ is \emph{weakly \'etale} if $f$ is flat and $\Delta_f$ is flat.
\end{dfn}

\begin{ex}
	\'Etale morphisms are weakly \'etale.
\end{ex}

\begin{ex}
	If $R \to S$ is a ring homomorphism, and $S = \colim_i S_i$ is a filtered colimit of \'etale $R$-algebras $S_i$, then $\Spec S \to \Spec R$ is weakly \'etale.
	The converse is not true (there are fun counterexamples).
\end{ex}

Bhatt-Scholze's pro-\'etale site $X_\proet$ (for $X$ a scheme) is the category of schemes weakly \'etale over $X$, where coverings are fpqc-coverings.
Bhatt-Scholze show that, if $\ell \in X_\proet$, there exists a sheaf $\ul{\ZZ_\ell}$ on $X_\proet$ such that $H^i(X, \ZZ_\ell) = H^i(X_\proet, \ul{\ZZ_\ell})$.

Our goal is to give a \emph{functorial} characterization of weakly \'etale morphisms.

\subsection{Lifting properties}

We recall a few important definitions:

\begin{dfn}
	A morphism $f: X \to Y$ is \emph{formally \'etale} if, for all solid diagrams
	\[
  \begin{tikzcd}
		\Spec A / I \rar \dar[hook] & X \dar["f"] \\
		\Spec A \rar \ar[ur, dashed] & Y,
  \end{tikzcd}
	\]
	where $I \subset A$ is a nilpotent ideal, there exists a unique dashed arrow making the diagram commute.
\end{dfn}

\begin{dfn}
	If $A$ is a ring, an ideal $I \subset A$ is \emph{henselian} if, for all solid diagrams
	\[
  \begin{tikzcd}
		\Spec A / I \rar \dar[hook] & X \dar \\
		\Spec A \rar \ar[ur, dashed] & Y,
  \end{tikzcd}
	\]
	with $X$ and $Y$ affine, there exists a unique dashed arrow making the diagram commute.
\end{dfn}

This is equivalent to other characterizations of henselianness, e.g.\ $I$ is contained in the Jacobson radical of $A$ and Hensel's lemma holds.

\begin{ex}
	If $I \subset A$ is nilpotent, then $(A, I)$ is nilpotent.
\end{ex}

\begin{dfn}
	A morphism $f: X \to Y$ has the \emph{unique henselian lifting property} if, for all solid diagrams
	\[
  \begin{tikzcd}
		\Spec A / I \rar \dar[hook] & X \dar \\
		\Spec A \rar \ar[ur, dashed] & Y,
  \end{tikzcd}
	\]
	with $(A, I)$ henselian, there exists a unique dashed arrow making the diagram commute.
\end{dfn}

\begin{rmk}
	The speaker expects that this process eventually stabilizes.
\end{rmk}

\begin{ex}
	An \'etale morphism of affine schemes has HLP.
\end{ex}

The main theorem of this talk is the following:

\begin{thm}
	A morphism $f: X \to Y$ is weakly \'etale if and only if $f$ has HLP.
\end{thm}

\subsection{Proof}

To prove the theorem, we should first understand some properties of HLP morphisms:
\begin{enumerate}
	\item A composite or base change of HLP morphisms is HLP.
	\item If $f: X \to Y$ and $g: Y \to Z$ are such that $g \circ f$ and $g$ satisfy HLP, then $f$ satisfies HLP.
	\item An open immersion satisfies HLP.
	\item For every pair $(A, I)$ with $I \subset A$ an ideal, there exists a universal \emph{henselization} map $(A, I) \to (A^h, I^h)$ with $(A^h, I^h)$ henselian.
		The ring map $A \to A^h$ is a filtered colimit of \'etale $A$-algebras.
		The ideal $A^h$ is equal to $I A^h$.
		The map $A / I \to A^h / I A^h$.\footnote{These last few properties are analogous to properties of the completion.}
\end{enumerate}

\begin{proof}[Proof of theorem]
	($\Leftarrow$). Suppose $f$ has HLP.
	Then $\Delta_f$ has HLP by properties (1) and (2).
	Thus it suffices to show that, if $f$ has HLP, then $f$ is flat.
	Without loss of generality we may assume $X$ and $Y$ are affine, so $f$ corresponds to a ring homomorphism $R \to S$.
	Let $P = R[x_i]_{i \in I}$ be a (possibly infinite type) polynomial $R$-algebra which surjects onto $R$, and let $(A, I)$ be the henselization of $(P, \ker(P \to S))$.
	We end up considering the diagram
	\[
  \begin{tikzcd}
		A/I & S \lar["\sim"] \\
		A \uar & R \uar.
  \end{tikzcd}
	\]
	The henselian lifting property gives a map $S \to A$, so $S$ is a direct summand of the free $R$-module $A$.
	Thus $S$ is flat.

	($\Rightarrow$) This is harder, so we'll only sketch part of it.
	To see why HLP is Zariski local on the source, suppose $X = \cup U_i$ where each $U_i \to Y$ has HLP.
	We seek to show $X$ has HLP.
	Given the HLP diagram, we get open covers $\Spec(A / I) = \cup_i D(f_i)$ and dashed arrows $\Spec A_f^h \to X$.
	By uniqueness of the dashed arrows, the two morphisms $\Spec A_{f_i f_j}^h \to X$ (for all $i, j$) agree.
	We can prove a sheaf property (using Gabber's affine analogue of proper base change) to glue the morphisms $\Spec A_f^h \to X$ to form a morphism $\Spec A \to X$.
\end{proof}

\section{3/4 (Hannah Larson) -- Tautological Classes on Strata of Differentials}

This is based on joint work with Dawei Chen.

\subsection{Background}

Let $C$ be a smooth curve of genus $g$.
The space of differentials on $C$ is $H^0(\omega_C)$, a vector space of dimension $g$.

These spaces assemble into a vector bundle $\Ec \to \Mfr_g$, called the \emph{Hodge bundle}.
We can (fiberwise) projectivize this to get a $\PP^{g-1}$-bundle $\PP \Ec \to \Mfr_g$.
The fiber over $C$ can be identified with the space of canonical divisors $D$ (i.e.\ $D \sim K$), i.e.\ $\PP \Ec$ is the moduli of such divisors.
These divisors are given by hyperplane sections of the canonical embedding $C \hookrightarrow \PP^g$.

\begin{ex}
	For $g = 3$, we have $\deg \omega_C = 2g - 2 = 4$.
	A general differential will have $4$ distinct zeroes, all of multiplicity one.
	However, there are higher codimension loci of divisors which have higher multiplicity in the zeroes.
	These correspond to flex points / bitangents / ... of the canonical embedding.
	The speaker drew a beautiful picture which I will not attempt to reproduce here.

	The strata of differentials with zeroes of multiplicity $\{ \mu_1, \dots, \mu_n \}$ has codimension $\sum_i (\mu_i - 1)$.
	One can use enumerative geometry to understand the fibers of these strata over $\Mfr_g$.
	
	Over the hyperelliptic locus in $\Mfr_g$, interesting phenomena arise when all $\mu_i$ are even.
	In this case, we consider $h^0(C, \frac{1}{2} D)$, where $\frac{1}{2} D$ is the $\theta$-characteristic.
	It turns out that the parity of $h^0(C, \frac{1}{2} D)$ is a deformation invariant, so we can use this value as an extra label for the strata.
\end{ex}

We also like to consider an \emph{ordered} version of the strata of differentials.

\begin{dfn}
	Given a partition $\mu_1, \dots, \mu_n$ of $2g - 2$, we define 
	\[
		P(\mu) = \bset{(C, z_1, \dots, z_n)}{\sum_i \mu_i z_i \sim K_C} \subset \Mfr_{g,n}.
	\]
\end{dfn}

To understand the Chow ring of the unordered version, we take $G$-invariants, where $G$ is the group that exchanges zeroes of the same multiplicity.

\subsection{Tautological classes}

The universal curve $\pi: \Cfr \to P(\mu)$ comes with sections $\sigma_1, \dots, \sigma_n: P(\mu) \to \Cfr$.
Let $\Zfr_i$ be the image of $\sigma_i$.
We can construct \emph{tautological classes} in $A^i(P(\mu))$:
\begin{itemize}
	\item $\psi_i = c_1(\sigma_i^* \omega_\pi) \in A^1(P(\mu))$.
	\item $\kappa_i = \pi_*(c_1(\omega_\pi)^{i+1}) \in A^i(P(\mu))$
	\item The line bundle $\Lc = \omega_\pi(-\sum_i \mu_i \Zfr_i)$ is fiberwise trivial, so by cohomology and base change, $\pi_* \Lc$ is a line bundle and $\pi^* \pi_* \Lc \xrightarrow{\sim} \Lc$.
		One can show that $\pi_* \Lc$ is the pullback of $\Oc_{\PP \Ec}(-1)$ to $P(\mu)$.
		We define $\eta = c_1(\pi_* \Lc) \in A^1(P(\mu))$.
\end{itemize}

\begin{dfn}
	The \emph{tautological ring} $R^*(P(\mu))$ is the subring of $A^*(P(\mu))$ generated by the $\psi_i$, $\kappa_i$, and $\eta$.
\end{dfn}

From now on we work with rational coefficients throughout.

\begin{prop}[Chen]
	$R^*(P(\mu))$ is generated by $\eta$.
	In fact, $\eta = (\mu_i + 1) \psi_i$ and $\kappa_j = \sum_i \big((\mu_i + 1)^j - 1\big) \psi_i^j$ (which is equivalent to a constant times $\eta^j$).
\end{prop}

\begin{proof}[Sketch of proof]
	We have 
	\[
		\pi^* \eta = c_1(\omega_\pi) - \sum_j \mu_j [\Zfr_j] \in A^1(\Cfr).
	\]
	Thus
	\begin{align*}
		\eta &= \sigma_i^* \pi^* \eta \sum_j \mu_j \sigma_j^* [\Zfr_j] \\
		     &= \psi_i - \sum_j \mu_j \sigma_j^* \sigma_{i*} [P(\mu)] \\
		     &= \psi_i + \sum_j \mu_j \delta_{ij} c_1(N_{\Zfr_i / \Cfr}) \\
		     &= (\mu_i + 1) \psi_i.
	\end{align*}
\end{proof}

\begin{cor}
	$R^*(P(\mu)) = \QQ[\eta] / (\eta^{e(\mu)})$.
\end{cor}

What is the minimal power $e(\mu)$?
There is a relation among $\kappa$-classes on $\Mfr_g$ in codimension $\left\lfloor g / 3 \right\rfloor - 1$.
Understanding the behavior of this relation would give a bound on the minimal $e(\mu)$.

\begin{conj}
	$e(\mu) \leq \left\lfloor g/3 \right\rfloor + 1$.
\end{conj}

The following is probably true, but details need to be worked out.

\begin{thm}[Chen-L]
	If $g \gg \sum_i \mu_1 + \dots + \mu_k$, then $e(\mu_1, \dots, \mu_k, 1, \dots, 1) \geq \left\lfloor g/3 \right\rfloor + 1$.
\end{thm}

The end goal is:

\begin{conj}
	If $g > C (\mu_1 + \dots + \mu_k)$, then $R^*(P(\mu_1, \dots, \mu_k, 1, \dots, 1)) = \QQ[\eta] / (\eta^{\left\lfloor g/3 \right\rfloor + 1})$.
\end{conj}

\section{3/11 (Chris O'Neill) -- Classifying Numerical Semigroups using Polyhedral Geometry}

\subsection{Numerical semigroups and Ap\'ery sets}

By a \emph{numerical semigroup} we mean a semigroup $S \subset (\NN, +)$ such that $\NN \setminus S$ is finite.
This last condition is equivalent to the generators of $S$ not sharing a single common factor.

\begin{ex}
	The \emph{McNugget semigroup} $S = \angles{6, 9, 20}$ (so named because it was previously the semigroup of possible Chicken McNugget orders) is numerical.
	43 is the largest natural number not contained in $S$.
\end{ex}

The \emph{multiplicity} of a numerical semigroup $S$ is $m = m(S) = \min(S \setminus 0)$.
The \emph{Ap\'ery set} of a numerical semigroup is $\operatorname{Ap}(S) = \bset{n \in S}{n - m \not\in S}$.
This can be written $\{ 0, a_1, a_2, \dots, a_{m - 1}\}$ where $a_i$ is the smallest element of $S$ such that $a \equiv i \pmod(m)$.
Many numerical properties of $S$ can be read from the Ap\'ery set.
However, computing the Ap\'ery set is difficult in general.

\begin{ex}
	The Ap\'ery set of the McNugget semigroup is $\{ 0, 49, 20, 9, 40, 29 \}$.
\end{ex}

\subsection{Kunz cones}

Suppose one is given a list of nonnegative integers $\{ 0, a_1, \dots, a_{m-1} \}$.
How can one tell whether this is the Ap\'ery set of some numerical semigroup?
We must have $a_i \equiv i \pmod m$ (after choosing some order).
However, there is one further condition to impose: we must have $a_i + a_j \geq a_{i + j}$ (where addition in subscripts is performed modulo $m$).
It turns out that this is enough.

\begin{thm}[Kunz]
	A subset $A = \{ 0, a_1, \dots, a_{m-1} \} \subset \NN$ is an Ap\'ery set if and only if $a_i \equiv i \pmod m$ for all $i$ and 
	\[
		a_i + a_j \geq a_{i+j} \textrm{ for } i + j \neq 0 \in \ZZ/m.
	\]
\end{thm}

The inequalities in the above theorem define the \emph{Kunz cone} $C_m \subset \RR^{m-1}$.
Integer points of the Kunz cone classify the possible Ap\'ery sets of cardinality $m$.
Thus numerical semigroups are classified by integer points of $C_m$ as $m$ varies.
We can ask:

\begin{qn}
	When do two semigroups $S, S' \in C_m$ lie in the relative interior of the same face of $C_m$?
\end{qn}

We can answer this by enhancing $\operatorname{Ap}(S)$ to a poset: say $a_i \leq a_j$ if $a_j - a_i \in S$.
If we replace $a_i$ by the equivalence class $[i] \in \ZZ/m$, we get a \emph{Kunz poset} (with underlying set $\{[0], \dots, [m - 1]\}$).

\begin{thm}[Kunz; Bruns, Garcia-Sanchez, O'Neill, Wilbourne]
	Two elements $S, S'$ lie in the same face of $C_m$ if and only if they have identical Kunz posets.
\end{thm}

Given a face $F$ of $C_m$, we can recover $\dim F$ from the Kunz poset of $S \in F$.
However, there are some faces of $C_m$ which do not contain numerical semigroups.
These turn out to exist for a good geometric reason.

\begin{dfn}
	Let $S$ and $S'$ be numerical semigroups, and let $a \in S', b \in S$ be coprime and not minimal generators.
	We call $T = aS + bS'$ the \emph{gluing} of $S$ and $S'$ by $a$ and $b$.
\end{dfn}

\begin{dfn}
	A numerical semigroup is called \emph{complete intersection} if it is built from one-generator numerical semigroups via gluings.
\end{dfn}

These turn out to have a useful geometric interpretation.

\section{3/18 (Joe Harris) -- Rational points on algebraic curves}

\subsection{Faltings' theorem and some related questions}

The starting point for this discussion is:

\begin{thm}[Faltings]
	Let $X$ be a curve of genus $g \geq 2$ over a number field $k$.
	Then $|X(k)| < \infty$.
\end{thm}

Some natural followup questions arise:

\begin{qn}
	Fix the number field $k$ and genus $g$.
	Is $|X(k)|$ bounded?
\end{qn}

On a hyperelliptic curve of genus $g$, we obtain $2g + 2$ rational points.
Thus we must fix $g$ when asking this question.

The original expectation was that this question had a negative answer.
However, recent work makes it seem more likely that the answer is affirmative.

\begin{qn}
	What should be the analogue of Faltings' theorem for varieties of higher dimension?
\end{qn}

\begin{conj}[Lang-Vojta]
	If $X$ is a variety of general type over a number field $k$, then $X(k)$ is not Zariski dense.
\end{conj}

A variant of the Lang-Vojta conjecture says that, if $X$ is hyperbolic (i.e.\ every map $\AA^1_\CC \to X_\CC$ is constant), then $|X(k)|$ is finite.
We'll instead focus on the following, which has a more geometric component. 

\begin{conj}[Strong Lang-Vojta]
	If $X$ is a variety of general type (in characteristic zero?), then all subvarieties of $X$ which are not of general type are contained in a common proper subscheme $Z \subsetneq X$ (which we can take to be the closure of the union).
	Furthermore, if $X$ is defined over a number field $k$, then $|(X \setminus Z)(k)| < \infty$.
\end{conj}

\subsection{Relationship between the conjectures}

It turns out that the questions we've been asking about Faltings' theorem are related!

\begin{thm}[Caporaso-Harris-Mazur]
	\begin{enumerate}
		\item The original Lang-Vojta conjecture implies boundedness of rational points for curves.
		\item The strong Lang-Vojta conjecture implies a stronger version of boundedness: there exists $N = N(g)$ such that, for all number fields $k$, there exist only finitely many curves $X$ of genus $g$ over $k$ such that $|X(k)| > N$.
	\end{enumerate}
\end{thm}

To understand this, let's consider an example.

\begin{ex}
	Fix a number field $k$, and let $F$ and $G$ be general quartic polynomials in $k[x, y, z]$.
	Look at the pencil $C_t := V(t_0 F + t_1 G) \subset \PP^2_k$ (for $[t_0 : t_1] \in \PP^1_k$), and let $\Cc \subset \PP^1_k \times \PP^2_k$ be the total space of this family.
	The projection $\Cc \to \PP^2$ is a blowup at 16 points, so $\Cc$ has many rational points.
	But we'd like to show that each $C_t$ has only finitely many rational points!

	To do this, let's consider $\Cc \times_{\PP^1} \Cc \subset \PP^1 \times \PP^2 \times \PP^2$.
	As $\Cc \subset \PP^1 \times \PP^2$ has bidegree $(1, 4)$ and $K_{\PP^1 \times \PP^2}$ has bidegree $(-2, -3)$, we see that $K_\Cc$ has bidegree $(-1, 1)$.
	By contrast, we can compute that $K_{\Cc \times_{\PP^1} \Cc}$ has tridegree $(0, 1, 1)$, so $\Cc \times_{\PP^1} \Cc$ is of general type.
	Thus, by Lang-Vojta, $\Cc \times_{\PP^1} \Cc$ all rational points of $\Cc \times_{\PP^1} \Cc$ should lie in a proper subvariety $Z$.
	By considering the projection $\Cc \times_{\PP^1} \Cc \to \Cc$, we obtain the desired bound.
\end{ex}

To show the theorem, if $X \to B$ is a family of varieties with ``maximal variation,'' then for $n \gg 0$, the variety $X^{\times_B n}$ is of general type.
We can apply this to the moduli space of curves.
Considering projections shows the result.

\section{4/1a (Shubham Sinha) -- Quantum $K$-invariants of Grassmannians via Quot schemes}

This is based on joint work with Ming Zhang.
Some notation:
\begin{itemize}
	\item We write $P_{r,k}$ for the set of length $r$ integer partitions $\lambda = (\lambda_1, \dots, \lambda_r)$ of $k$.
	\item If $\lambda \in P_{r,k}$, we get a complement partition $\lambda^* = (k - \lambda_r, \dots, k - \lambda_1)$.
	\item For a partition $\lambda$, we define a Schur polynomial $s_\lambda(x_1, \dots, x_r)$ using determinants.
		Products of these are given by the Littlewood-Richardson rule.
\end{itemize}

\subsection{Vector bundles on $\Gr(r, n)$}.

Let $\Gr(r, n)$ be the Grassmannian of rank $r$ subspaces of $\CC^n$.
This has many interesting vector bundles:
\begin{itemize}
	\item Trivial vector bundles
	\item The universal subbundle $\Sc \subset \Gr(r, n) \times \CC^n$ and the universal quotient $\Gr(r, n) \times \CC^n \twoheadrightarrow \Qc$.
	\item Symmetric / exterior powers of $\Sc$, $\Qc$, $\dots$
	\item \emph{Schur bundles}: for any partition $\lambda = (\lambda_1, \dots, \lambda_r)$, a bundle $\SS^\lambda(\Sc)$ with fibers given by $\SS^\lambda(\CC^r)$, the irreducible $\GL_r$-representation with weights $(\lambda_1, \dots, \lambda_r)$.
\end{itemize}

By the Borel-Weil-Bott theorem, we have the following:

\begin{thm}
	For any nonempty $\lambda \in P_{r,n-r}$, we have:
	\begin{enumerate}
		\item $H^i(\Gr(r, n), \SS^\lambda(\Sc)) = 0$ for all $i \geq 0$.
		\item $H^0(\Gr(r, n), \SS^\lambda(\Sc^\vee)) = \SS^\lambda(\CC^n)$.
		\item $H^i(\Gr(r, n), \SS^\lambda(\Sc^\vee)) = 0$ for all $i \geq 0$.
	\end{enumerate}
\end{thm}

This can be used to understand $\Db(\Gr(r, n))$, but we will not need this for the talk.

\begin{cor}
	\begin{enumerate}
		\item $\chi(\Gr(r, n), \SS^\lambda(\Sc)) = 0$.
		\item $\chi(\Gr(r, n), \SS^\lambda(\Sc^\vee)) = s_\lambda(1, \dots, 1) = \dim \SS^\lambda(\CC^n)$.
	\end{enumerate}
\end{cor}

\subsection{Quot schemes}

Let $C$ be a smooth projective curve.
Let $\Mor_d(C, \Gr(r, n))$ be the space of degree $d$ morphisms $f: C \to \Gr(r, n)$.
Note that	$\Mor_d(C, \Gr(r, n))$ consists of short exact sequences
\[
  \SES{E}{\Oc_C^{\oplus n}}{F}
\]
such that $E$ and $F$ are vector bundles on $C$, $\rk E = r$, and $\deg E = -d$.
In general, this moduli space is only quasiprojective, but we can compactify it by embedding it into the \emph{Quot scheme} $\Quot_d(C, r, n)$.
This is defined using short exact sequences as above, but $E$ and $F$ are only required to be \emph{coherent sheaves} rather than vector bundles.
Much is known about virtual intersection theory on these Quot schemes.

On $C \times \Quot_d$, there is a universal SES
\[
	\SES{\Ec}{\Oc^{\oplus n}}{\Fc}
\]
which we can use to understand the ``evaluation map'' of the Quot scheme.
More precisely, if we view $\Gr(r, n)$ as a GIT quotient $M_{r,n}^{\mathrm{ss}} / \GL_r$, we have a commutative square
\[
	\begin{tikzcd}
		\Mor_d(C, \Gr(r, n)) \rar["\ev_p"] \dar[hook] & \Gr(r, n) \dar[hook] \\
		\Quot_d(C, r, n) \rar["\tilde{\ev}_p"] & [M_{r,n} / \GL_r].
	\end{tikzcd}
\]
Pulling back the standard representation along $\tilde{\ev}_p$ gives $\Ec_p := \Ec|_{p \times \Quot}$.

\subsection{Formulas}

\begin{thm}[Sinha-Zhang '24]
	Let $\lambda \neq \varnothing$.
	Then:
	\begin{enumerate}
		\item $\chi(\Quot_d(\PP^1, r, n), \SS^\lambda(\Ec_p)) = 0$.
		\item $\chi(\Quot_d(\PP^1, r, n), \SS^\lambda(\Ec_p^\vee)) = [t^d] s_{\lambda + (d^r)}(z_1, \dots, z_n)$ where $z_1, \dots, z_n$ are the distinct roots of $p(z) = (z - 1)^n + (-1)^r t z^{n-r}$ and $\lambda + (d^r) = (\lambda_1 + d, \dots, \lambda_r + d)$.
	\end{enumerate}
\end{thm}

\begin{ex}
	When $d = 0$, this recovers the Borel-Weil-Bott results for $\Gr(r, n)$.
\end{ex}

\begin{ex}
	\[
		\sum_{d \geq 0} q^d \chi(\Quot_d(\PP^1, r, n), \wedge^m \Ec_p) = \begin{cases}
			\binom{n}{m} \frac{1}{1 - q} & m < r \\
			\binom{n}{r} \frac{1}{(1 - q)^2} & m = r.
		\end{cases}
	\]
\end{ex}

\begin{conj}
	$H^i(\Quot, \SS^\lambda(\Ec_p)) = 0$ for $i \geq 0$.
\end{conj}

\begin{qn}
	Are there formulas for higher genus?
\end{qn}

\begin{ex}
	We have $\chi(\Quot_d(Cc, r, n), \SS^\lambda(\Ec_p^\vee)) = 0$ for $\lambda$ strictly contained in the box of size $(r, g)$ and $d \gg 0$.
\end{ex}

\subsection{Quantum $K$-theory}

Many of the genus $0$ formulas here can be used to recover results for the quantum $K$-ring of $\Gr(r, n)$.
This quantum $K$-ring can be related to $\Quot$ using explicit wall-crossing.
For genus $g > 0$, the results involve TQFT.

To understand the quantum $K$-ring, we first introduce some notation.
For $V \in K(M_{r,n} / \GL_r)$, define $\angles{\angles{V}}_0^{\Quot} = \sum_{d \geq 0} q^d \chi(\Quot, \tilde{\ev}_p^* V)$.

We have 
\[
	K(\Gr(r,n)) = \spanop \bset{\SS^\lambda(\Sc)}{\lambda \in P_{r,n-r}} = \spanop \bset{\Oc_\lambda}{\lambda \in P_{r,n-r}}
\]
where $\Oc_\lambda$ is the structure sheaf of a Schubert cycle.
The product of the $\Oc_\lambda$'s are given by the K-theoretic Littlewood-Richardson rule.
The quantum $K$-theory $QK(\Gr(r, n))$ is $K(\Gr(r, n)) \otimes \ZZ[[q]]$, but the product is deformed.
The structure coefficients $N_{\lambda\mu}^\nu$ in this product can be defined via stable maps and Gromov-Witten theory.

\begin{thm}[Sinha-Zhang]
  $N_{\lambda\mu}^\nu = \angles{\angles{\tilde{\Oc}_\lambda \cdot \tilde{\Oc}_\mu \cdot \tilde{\Oc}_\nu^*}}_0^{\Quot}$.
\end{thm}

This formula is easier to work with!

\section{4/1b (Brian Yang) -- Young Tableaux Combinatorics in the Springer Resolution}

Springer theory gives a connection between algebraic geometry, combinatorics, and representation theory.

\subsection{The Springer resolution}

We are interested in studying the \emph{characteristic polynomial map} $\chi: \slfr_n \to \cfr$, where $\cfr = \slfr_n \sslash W \simeq \AA^{n-1}$.
The map $\chi$ is not smooth: notably, $\Nc = \chi\inv(0)$ is singular.
We can resolve this as follows.
Let $\Bc$ be the flag variety, and consider the \emph{Grothendieck-Springer space}
\[
	\tilde{\slfr}_n = \bset{(x, b) \in \slfr_n \times \Bc}{x \textrm{ preserves } b}.
\]
This admits a map $\tilde{\chi}: \tilde{\slfr}_n \to \hfr$, where $\hfr$ is the canonical Cartan.
Let $\tilde{\Nc} = \tilde{\chi}\inv(0)$.

We obtain a commutative diagram
\[
	\begin{tikzcd}
		\tilde{\Nc} \rar \dar["p"] & \Bc \rar["\tilde{\chi}"] \dar & \hfr \dar \\
		\Nc \rar & \slfr_n \rar["\chi"] & \cfr.
	\end{tikzcd}
\]
The leftmost vertical map $p$ is the \emph{Springer resolution}.
For $x \in \Nc$, the preimage $p\inv(x)$ is the \emph{Springer fiber} (consisting of flags stabilized by $x$).
The Springer fiber depends only on the Jordan type $\lambda$ of $x$, so we will write $\Bc_\lambda$ for the Springer fiber.

\begin{ex}
	If $\lambda = (1, \dots, 1)$ (so $x = 0$), then $\Bc_\lambda = \Bc$.
\end{ex}

\begin{ex}
	For $\lambda = (n)$ (so $x$ has one Jordan block and $0$ has geometric multiplicity $1$), then $\Bc_\lambda$ is a point.
\end{ex}

\begin{ex}
	For $\lambda = (n-1, 1)$ (corresponding to a subregular nilpotent element), $\Bc_\lambda$ is a chain of $n - 1$ copies of $\PP^1$.
\end{ex}

\begin{ex}
	If $n = 4$ and $\lambda = (2, 2)$, $\Bc_\lambda$ is the union of $\PP^1 \times \PP^1$ and a Hirzebruch surface $F_2$.
\end{ex}

\subsection{Young tableaux}

Let $\lambda = (\lambda_1 \geq \lambda_2 \geq \dots \geq \lambda_ell)$ be a partition of $n$.
A (standard?) \emph{Young tableau} of shape $\lambda$ is a filling of the cells in the ``reverse staircase'' corresponding to $\lambda$ which increases along rows and columns.
Let $\SYT(\lambda)$ be the set of standard Young tableaux of shape $\lambda$.

We can define a map $\Bc_\lambda \to \SYT(\lambda)$ as follows.
Given $x \in \Nc$ of Jordan type $\lambda$, pick a flag $V_\bullet \in \Bc_x$.
Let $\mu^{(i)}$ be the Jordan type of $x$ acting on $V_n / V_{n-i}$.
Then we obtain a chain of increasing Young diagrams $0 = \mu^{(0)} \subset \mu^{(1)} \subset \dots \subset \mu^{(n)} = \lambda$ where each differs from the next by adding a single cell.

For $P \in \SYT(\lambda)$, let $\Bc_P$ be the preimage of $P$ in $\Bc_\lambda$.

\begin{thm}
	For each $P \in \SYT(\lambda)$, $\Bc_P$ is an irreducible locally closed subvariety of $\Bc_\lambda$ of dimension $\sum_i (i - 1) \lambda_i$.
	This dimension is independent of $P$, and the closures $\ol{\Bc}_P$ are exactly the distinct irreducible components of $\Bc_\lambda$.
\end{thm}

\begin{ex}
	For the subregular nilpotent orbit, the $\Bc_P$ correspond to different labelings of the bottom cell in the staircase diagram for $\lambda$.
\end{ex}

To prove the above theorem, one writes $\Bc_P$ as an iterated fiber bundle $\Bc_P = Y_n \to Y_{n-1} \to \dots \to Y_0 = \pt$.
The fibers of each $Y_{i+1} \to Y_i$ can be viewed as open subsets of a given projective space, and the rank of this map is exactly $r - 1$ where $r$ is the row in $P$ containing entry $n - i$.

\subsection{The nilpotent Steinberg variety}

The \emph{nilpotent Steinberg variety} is
\[
	\St = \tilde{\Nc} \times_\Nc \tilde{\Nc}.
\]
We have two ways of describing the irreducible components of $\St$.

The first is via the Bruhat decomposition.
For $b_1, b_2 \in \Bc$, there is a basis $e_1, \dots, e_n$ of $V = \CC^n$ such that $b_1$ is the standard flag and $b_2$ is the flag corresponding to a permutation of $e_1, \dots, e_n$.
Let $w$ be the permutation here.
The \emph{relative position} of $b_1$ and $b_2$ is given by $w \in S_n$.

\begin{thm}[Bruhat decomposition]
	There is a decomposition $\St = \sqcup_{w \in S_n} \St_w$, where $\St_w = \bset{(x, b_1, b_2)}{b_1, b_2 \textrm{ in rel.\ pos.} w}$.
	Moreover, $\dim \St_w = \dim \St = 2 \dim \Bc$.
\end{thm}

\begin{cor}
	The closures $\ol{\St}_w$ are the irreducible components of $\St$.
\end{cor}

The second way to describe the irreducible components of $\St$ is as follows.
Let $\lambda$ be a Jordan type, and let $P, Q \in \SYT(\lambda)$.
The closures of $\St \cap (\Oc_\lambda \times \Bc_P \times \Bc_Q)$ are the irreducible components of $\St$.

We end up seeing that the following are in bijection:
\begin{enumerate}
	\item Elements of $S_n$,
	\item Irreducible components of $\St$, and
	\item Pairs of standard Young tableaux of size $n$ and the same shape.
\end{enumerate}

These can be related via the Robinson-Schensted correspondence, but at this point it was 6pm and I had to leave the seminar.

\section{4/8 (Kabir Kapoor) -- The ADHM Construction}

Many interesting physical theories are \emph{gauge theories}, based on groups:

\begin{itemize}
  \item Electromagnetism is a $\mathop{U}(1)$-gauge theory.
  \item The weak nuclear force is a $\mathop{SU}(2)$-gauge theory.
  \item The strong nuclear force is a $\mathop{SU}(3)$-gauge theory.
\end{itemize}

\subsection{Review of electromagnetism}

Let's say our spacetime is $M^4 = \RR^3 \times \RR^1$.
On this, we have an \emph{electric field} $\Ebf$ and a \emph{magnetic field} $\Bbf$.
In vacuum, these satisfy \emph{Maxwell's equations}
\begin{align*}
	\nabla \cdot \Ebf &= 0 \\
	\nabla \times \Ebf + \partial_t \Bbf &= 0 \\
	\nabla \cdot \Ebf &= 0 \\
	\nabla \times \Ebf - \partial_t \Bbf &= 0.
\end{align*}

Applying $\partial_t$ again, we get the \emph{wave equations}
\begin{align*}
	\partial_t^2 \Ebf &= \nabla^2 \Ebf \\
	\partial_t^2 \Bbf &= \nabla^2 \Bbf.
\end{align*}
Letting $\vec{\Ec} = \Ebf + i \Bbf$, we have wave solutions
\[
	\vec{\Ec}(t, \vec{x}) = \vec{\Ec}_0 \exp(-i(\abs{\vec{k}} t - \vec{k} \cdot \vec{x})).
\]

Let's reinterpret $\Bbf$ as a 1-form $B$ and $\Ebf$ as a 2-form $E$.
Let $F = E + B \wedge dt$.
Then Maxwell's equations can be rewritten as
\[
	d F = 0, \,\,\, d \star F = 0.
\]
Here $\star: \Omega^2(M) \to \Omega^2(M)$ satisfies $\star^2 = -1$.
We have a decomposition $F = F_+ + F_-$ into eigenvectors for $\star$.
These $F_+$ and $F_-$ are known as \emph{instantons}.

If $F = dA$, the equation $dF = 0$ holds for free.
When we can write $F = dA$, $A$ is non-unique: it can be replaced by $A + d\phi$ for any $\phi$.

To make sense of this geometrically, let's consider a principal $G$-bundle $\pi: P \to M$.
A choice of trivialization is given by a section $s: M \to P$.
Any other trivialization can be written as $s' = s \cdot g$ for some $g: M \to G$.
(This is also true locally.)

A connection is given by an element $\omega \in \Omega^1(P; \gfr)$, and the pullback $s^* \omega$ lives in $\Omega^1(M; \gfr)$.
Changing our trivialization to $s' = s \cdot g$ gives $(s')^* \omega = g (s^* \omega) g\inv + dg \cdot g\inv$.
In the case of $G = \mathop{U}(1)$, if $s^* \omega = i A$ and $g(x) = e^{i \phi(x)}$, then $(s')^* \omega = i(A + d\phi)$.

\subsection{$\SU(2)$-instantons}

How do we classify $\SU(2)$-instantons on $\RR^4$ modulo gauge equivalence?
Really, we are interested in ``finite energy'' instantons, i.e.\ those that extend to $S^4$.
Atiyah and Ward proved that the following are equivalent:
\begin{itemize}
	\item $\SU(2)$-instantons on $S^4$.
	\item Holomorphic vector bundles $E \to \CC\PP^3$ together with an isomorphism $\Sigma: E \xrightarrow{\sim} E^\vee$ such that $\Sigma^2 = -1$ and the family is trivial when restricted to a specific large family of embedded $\CC\PP^1$'s.
\end{itemize}
This is controlled by a Hopf fibration $\CC\PP^1 \hookrightarrow \CC\PP^3 \twoheadrightarrow \HH\PP^1 = S^4$, where the last map is $[z_1 : z_2 : z_3 : z_4] \mapsto [z_1 + j z_2 : z_3 + j z_4]$.
The ``large family'' referenced above consists of the fibers in this Hopf fibration.

Work of Horrocks, Barth-Hulek, Beilinson, and Nakajima discussed the further classification of such objects.
We'd like to understand the obstructions to splitting a vector bundle as a direct sum of line bundles.
To this end, we introduce \emph{monads}.
A monad is a complex of vector bundles $A \xrightarrow{a} B \xrightarrow{b} C$, each of which splits as a direct sum of line bundles, such that $a$ is injective, $c$ is surjective, and $ca = 0$.
For a given vector bundle $E$, we like to consider monads with $E = \ker c / \im a$.

\begin{thm}
	Let $E$ be a rank $r$ vector bundle over $\PP^n$, and let $\Phi$ be a trivialization of $E$ over a line at infinity $\ell_\infty$.
	Then $E$ arises from a monad of the form $\Oc(-1)^{\oplus k} \to \Oc^{\oplus 2k + r} \to \Oc(1)^{\oplus k}$, where $k = c_2(E)$.
\end{thm}

This reduces the classification of such $E$ to a (hard) linear algebra problem!

\begin{proof}[Sketch of Beilinson's proof]
	Let $X = \PP^n$.
	Beilinson resolved the structure sheaf of the diagonal $\Delta \subset X \times X$ by locally free sheaves.
	Note that
	\[
		E = \Rbf \pi_{1*}(\Lbf \pi_2^* E \otimes \Oc_\Delta).
	\]
	Replacing $\Oc_\Delta$ by its resolution, we obtain a double complex.
	One of the spectral sequences obtained from this double complex clearly converges to $E$.
	The $E_1$ page of the other (which must converge to the same thing) gives the monad.
\end{proof}

Related problems also have interesting algebro-geometric descriptions.

\section{4/15 (Cameron Chang) -- Combinatorial Duality Theorems and Serre Duality}

This is an expository talk -- most of the material can be found in Fulton's book on toric varieties.

\subsection{Three polynomials appearing in combinatorics}

Here are a few classic polynomials in combinatorics.
\begin{enumerate}
	\item For a graph $G$, let $c(G, k)$ denote the number of proper $k$-colorings.
		Birkhoff proved that $c(G, k)$ is polynomial in $k$.
		We call $c(G, k)$ the \emph{chromatic polynomial} of $G$.
	\item For a poset $Q$, let $\Omega(Q, k)$ denote the number of order-preserving maps $Q \to [k]$.
		Stanley proved that this is polynomial in $k$.
		We call $\Omega(Q, k)$ the \emph{order polynomial} of $Q$.
	\item For a lattice polytope $P$, let $\ehr(P, k)$ be the number of lattice points of $kP$.
		Ehrhart proved that this is polynomial in $k$.
		We call $\ehr(P, k)$ the \emph{Ehrhart polynomial} of $Q$.
\end{enumerate}

These polynomials have interesting interpretations when evaluated at negative numbers.
\begin{enumerate}
	\item $\abs{c(G, -1)}$ counts the number of acyclic orientations of $G$.
	\item $(-1)^{|Q|} \Omega(Q, -k)$ counts the number of strictly order-preserving maps $Q \to [k]$.
	\item Ehrhart-Macdonald reciprocity tells us that $(-1)^{\dim P} \ehr(P, -k)$ computes the number of interior lattice points of $k P$.
\end{enumerate}

Given a poset $Q$, there is an \emph{order polytope} $P_Q$ such that $\Omega(Q, k) = \ehr(P_Q, k-1)$, and the number of interior lattice points of $m P_Q$ counts the number of strict order-preserving maps $Q \to [m - 1]$.
Thus Ehrhart-Macdonald reciprocity implies the claim about order polynomials.

We can also reduce the question about graphs to a question about posets, hence showing that Ehrhart-Macdonald reciprocity implies all of our desired claims.
We are still left wondering why Ehrhart-Macdonald reciprocity holds.

\subsection{Toric varieties}

\begin{dfn}
	A toric variety is a normal separated complex variety together with a dense open torus $T \subset X$ such that the action of $T$ on itself extends to an action of $T$ on $X$.
\end{dfn}

There is a remarkable bijection between lattice polytopes and ample $T$-invariant Cartier divisors on complete toric varieties.
We will use this to construct projective toric varieties from lattice poltyopes.
Recall that all toric varieties can be constructed via \emph{fans}.

\begin{dfn}
	A \emph{fan} $\Delta$ is a collection of cones $\sigma \subset \RR^n$ which is closed under taking faces and is such that $\sigma \cap \sigma'$ is a face of $\sigma$ and $\sigma'$ for all $\sigma, \sigma' \in \Delta$.
\end{dfn}

\begin{dfn}
	Given a lattice polytope $P$, the \emph{normal fan} $\Delta_P$ has cones (indexed by faces $F$ of $P$)
	\[
		\sigma_F := \bset{v \in \RR^n}{\ip{v}{u} \leq \ip{v}{u'} \textrm{ for all } u \in F, u' \in P}.
	\]
\end{dfn}

We can construct a dictionary between properties of lattice polytopes and properties of the corresponding toric varieties.
First fix some notation: let $v_1, \dots, v_k$ be the edges in the fan, and let $\sigma_1, \dots, \sigma_r$ be the maximal cones.
Note that $\cup_i \sigma_i = \RR^n$, so the toric variety $X_P$ is complete.

\begin{enumerate}
	\item $T$-invariant Cartier divisors $D$ on $X_P$ correspond to functions $\psi_D: \RR^n \to \RR$ such that $\psi_D|_{\sigma_i} = \ip{u_i}{-}$ for some $u_i \in \ZZ^n$.
	\item A line bundle $\Lc(D)$ is gbgs if and only if, for all $\sigma_i$, $\ip{u_i}{-} \geq \psi_D$ outside of $\sigma_i$.
	\item A line bundle $\Lc(D)$ is ample if and only if, for all $\sigma_i$, $\ip{u_i}{-} > \psi_D$ outside of $\sigma_i$, i.e.\ $\Lc(D)$ is ``strictly convex.''\footnote{In particular, ample implies gbgs for line bundles on toric varieties.}
		Such ``strictly convex support functions'' correspond to lattice polytopes $P_D$.
\end{enumerate}

Here are some useful facts about the lattice polytopes $P_D$:
\begin{enumerate}
	\item For $n \in \NN$, we have $P_{nD} = n P_D$.
	\item $h^0(X, \Lc(D))$ counts the number of lattice points in the polytope $P_D$.
	\item Higher cohomology of ample line bundles vanishes.
		In particular, we have $\chi(X, \Lc(kD)) = h^0(X, \Lc(kD)) = \ehr(P_D, k)$ for $k \geq 0$.
		Since $\chi(X, \Lc(kD))$ is a polynomial in $k$, it follows that the Ehrhart polynomial is in fact a polynomial.
\end{enumerate}

\subsection{Ehrhart-Macdonald duality via Serre duality}

For simplicity, we will assume our toric variety $X$ is smooth.
This implies that the corresponding lattice polytopes are \emph{simple}.
The canonical bundle $K_X$ then corresponds to the support function given by $\psi_K(v_i) = -1$ for all $i$.

One can show that $P_{D+K}$ is the interior of $P_D$ for any $D$.
Serre duality gives
\[
	(-1)^n \ehr(P_D, -k) = (-1)^n \chi(X, \Oc(-kD)) = \chi(X, \Oc(kD + K))
\]
which equals $h^0(X, \Oc(kD + K))$ by Kodaira vanishing.
Thus $(-1)^n \ehr(P_D, -k)$ counts the number of interior points of $P_{kD}$.

\section{4/22a (Lizzie Pratt) -- The Segre Determinant}

This was a very nice talk, but it was a slide talk, so I did not take notes.

\section{4/22b (Bernd Sturmfels) -- The Likelihood Correspondence}

This is based on joint work with T.\ Kahle, H.\ Schenck, and M.\ Wiesmann.

\subsection{Setup}

Suppose we are given homogeneous polynomials $f_1, \dots, f_m \in \CC[x_1, \dots, x_n]$ of degrees $d_1, \dots, d_m$.
We may interpret this as an arrangement $\Ac$ of hypersurfaces in $\PP^{n-1}$.
We may define
\begin{itemize}
	\item The \emph{likelihood function} $f_1^{s_1} f_2^{s_2} \dots f_m^{s_m}$ 
		This is a rational function on $\PP^n$ determined by integers $s_i$ such that $\sum_{i=1}^m d_i s_i = 0$.
	\item The \emph{log-likelihood function} $\ell_\Ac = s_1 \log(f_1) + \dots + s_m \log(f_m)$.
		Here the $s_i$ do not need to be integers.
\end{itemize}

\subsection{The problem}

The \emph{likelihood correspondence} $\Lc_\Ac$ is the Zariski closure in $\PP^{m-1} \times \PP^{n-1}$ of the set of all pairs $(\ol{s}, \ol{x})$ where $\ol{x}$ is a critical point of $\ell_\Ac$ for $\ol{s}$.
Write $I_\Ac \subset \CC[\ol{x}, \ol{s}]$ for the prime ideal of $\Lc_\Ac$.
We'd like to describe $I_\Ac$.

Note that the ``Euler relation'' $\sum_i d_i s_i$ is in $I_\Ac$.

The map $\Lc_\Ac \to \PP^{m-2} \subset \PP^{m-1}$ (where $\PP^{m-2}$ is cut out by the Euler relation) is $\chi$-to-one.
Here $\chi$ is:
\begin{itemize}
	\item The ``maximum likelihood degree'' of $\Ac$.
	\item The absolute value of the topological Euler characteristic of $\PP^{n-1} \setminus \Ac$.
	\item (In generic cases) the coefficient of $z^{n-1}$ in
		\[
			\frac{(1 - z)^n}{(1 - d_1 z) \dots (1 - d_m z)}.
		\]
\end{itemize}

\begin{ex}
	Let $n = 3$, $m = 4$, $d_1 = 1$, $d_2 = d_3 = d_4 = 2$.
	That is, we are looking at one line and three conics in the plane.
	The ML degree here is 13.
	We may write $\Lc_\Ac$ as a surface in $\PP^3 \times \PP^2$.
	The ideal $I_\Ac$ has six generators:
	\begin{itemize}
		\item One of degree $(1, 0)$,
		\item Three of degree $(1, 4)$, and
		\item Two of degree $(1, 5)$.
	\end{itemize}
\end{ex}

\subsection{Results}

We can construct a certain $(m + 1) \times (m + n - 1)$ matrix $Q^s_{\setminus 1}$ from the $f_i$, the $s_i$, and the $\partial f_i / \partial x_j$.

\begin{thm}
	Assume $m \geq n + 1$, $d_i \geq 2$ for all $i$, and $\Ac$ is strict normal crossing (SNC).
	Then $I_\Ac$ is minimally generated by the Euler relation and $\binom{m+n-1}{n-2}$ maximal minors of $Q^s_{\setminus 1}$.
	We can read off the $\ZZ^2$-degrees of all of these.
\end{thm}

There's a more general result that allows some $d_i$ to be $1$.
This gives some connections to the theory of hyperplane arrangements and toric models.

We should of course discuss the ``snc'' condition above.

\begin{thm}
	The algebraic GKZ discriminant $\nabla$ agrees with the topological Euler discriminant $\nabla_\chi$, i.e.\ the locus where $\abs{\chi(\PP^{n-2} \setminus \Ac)}$ drops below its generic value.
\end{thm}

\begin{ex}
	Continuing with our example above, $\nabla$ has 13 irreducible components and its $\ZZ^4$-degree is $(18, 25, 25, 25)$.
	The arrangement $\Ac$ is SNC if and only if all of the following happen:
	\begin{itemize}
		\item The conics are irreducible (giving 3 components?)
		\item There's no tangency (giving 6 components?)
		\item There are no triple intersections (giving 4 components?)
	\end{itemize}
\end{ex}

\subsection{The kernel of a matrix over a ring}

\begin{ex}
	Consider $R = \ZZ$.
	The kernel of the matrix
	\[
		M = \begin{bmatrix}
			6 & 10 & 15
		\end{bmatrix}
	\]
	can be computed using matroids.
	The \emph{circuits} are the maximal minors of
	\[	
		\begin{bmatrix}
			6 & 10 & 15 \\
			s_1 & s_2 & s_3
		\end{bmatrix}.
	\]
	The corresponding column vectors generate $\ker M$.
	This doesn't require any division!
\end{ex}

\begin{prop}
	The kernel of a matrix $M$ is generated by the circuits of $M$ provided that the ideal of maximal minors of $M$ contains a long enough regular sequence.
\end{prop}

\section{4/29b (Feiyang Lin) -- Tame Splitting Loci Have Rational Singularities}

\subsection{Setup}

Let $B$ be a scheme, and let $\Ec$ be a vector bundle of rank $r$ and fiberwise degree $d$ on $B \times \PP^1$.
A \emph{splitting type} is $\ebf = (e_1, \dots, e_r)$ with $e_1 \leq \dots \leq e_r$ and $\sum_i e_i = d$.
The \emph{splitting locus} $\Sigma_{\ebf}(\Ec)$ is $\bset{b \in B}{\Ec|_{\{b\} \times \PP^1} \cong \Oc(\ebf) := \oplus_i \Oc(e_i)}$.
The closure $\ol{\Sigma}_\ebf(\Ec)$ is equal to $\cup_{\ebf' \leq \ebf} \Sigma_{\ebf'}(\Ec)$.

We can understand this universally: let $\Bfr_{r,d}$ be the moduli stack of vector bundles of rank $r$ and degree $d$ on $\PP^1$.
Then there is a universal closed subvariety $\ol{\Sigma}_\ebf \subset \Bfr_{r,d}$ and a universal open subvariety $\Sigma_\ebf = B\Aut(\Oc(\ebf)) \subset \ol{\Sigma}_\ebf$.

Following Eisenbud-Schreyer, one can understand these in terms of schemes by considering the smooth covers $\AA^N = \Ext^1(\Oc(\ebf), \Oc(\ebf)) \to \Bfr_{r,d}$.

\begin{ex}
	The varieties $\ol{\Sigma}_{(e,d-e)}$ correspond to cones over higher secant varieties of the rational normal curve $C_{d-2} \subset \PP^{n-2}$.
\end{ex}

There are various reasons we are interested in studying these splitting loci:
\begin{itemize}
	\item For Hurwitz-Brill-Noether theory, the geometry of $W^r_d(C)$ for a general $k$-gonal curve of genus $g$ corresponds to certain splitting loci of rank $k$ and degree $d - g + 1 - k$.
	\item Splitting loci are related to determinantal loci.
  \item Splitting loci correspond to Harder-Narasimhan strata of $\mathrm{Bun}_{\GL_n}$.
\end{itemize}

It is known that in general $\ol{\Sigma}_\ebf$ is normal and Cohen-Macaulay.
However, an explicit resolution of singularities is missing in the literature.

\subsection{Resolving singularities}

\begin{dfn}
	Fix $r$, $\rbf = (r_1 \geq r_2 \geq \dots)$ with $r \geq r_1$, $d$, $\dbf = (d_1, \dots, d_m)$.
	Also fix $B$ and a vector bundle $\Ec$ on $\PP^1$.
	Define $\FQuot(B, \Ec; \rbf, \dbf)$ with $T$-points consisting of
	\begin{itemize}
		\item $f: T \to B$ together with
		\item $\tilde{f}^* \Ec \twoheadrightarrow \Qc_1 \twoheadrightarrow \dots \twoheadrightarrow \Qc_m$, such that:
		\item $\rk \Qc_i = r_i$, $\deg (\Qc_i)_t = d_i$, and $\Qc_i$ is flat over $T$.
	\end{itemize}
\end{dfn}

We may also view this as constructed via repeated construction of ``relative Quot schemes.''
Dually, instead of quotients, we may consider a sequence of inclusions $\Ec_1 \subset \dots \Ec_m \subset \tilde{f}^* \Ec$.

\begin{dfn}
	Given $\ebf$, say a sequence of injections $\Ec_1 \subset \dots \subset \Ec_m \subset \Oc(\ebf)$ is an \emph{admissible flag} if:
	\begin{itemize}
		\item Each $\Ec_{i+1} / \Ec_i$ is a balanced vector bundle, and
		\item Up to automorphisms of $\Ec_i$, the inclusion $\Ec_i \hookrightarrow \Ec_{i+1}$ is the unique such injection.
	\end{itemize}
\end{dfn}

\begin{dfn}
	A tuple $(\rbf, \dbf)$ is $\ebf$-admissible if there exists an admissible flag as above such that $r_i = \rk \Oc(\ebf) / \Ec_i$ and $d_i = \deg \Oc(\ebf) / \Ec_i$.
\end{dfn}

\begin{thm}[Lin]
	Let $(\rbf, \dbf)$ be $\ebf$-admissible.
	Let $\tilde{\Sigma}_{\ebf, \rbf, \dbf} = \FQuot(\Bfr_{r,d}, \Ec_\univ; \rbf, \dbf)$, and let $\rho: \tilde{\Sigma}_{\ebf, \rbf, \dbf} \to \ol{\Sigma}_{\ebf}$ be the natural map.
  Then $\tilde{\Sigma}_{\ebf, \rbf, \dbf}$ is smooth and irreducible of dimension $-r^2 - h^1 \mathcal{E}\!\mathop{nd}(\Oc(\ebf))$.
	Furthermore, $\rho\inv(\Sigma_{\ebf}) \xrightarrow{\sim} \Sigma_{\ebf}$.
\end{thm}

\begin{dfn}
	Say $\ebf$ is \emph{tame} is $\ebf = (\fbf, \gbf)$ where $\fbf$ and $\gbf$ are balanced.
	(Equivalently, there is an admissible flag $\Ec_1 \subset \Oc(\ebf)$.)
\end{dfn}

\begin{thm}[Lin]
	Let $k = \ol{k}$, $\charop k = 0$.
	If $\ebf$ is tame, then $\ol{\Sigma}_{\ebf}$ has rational singularities.
\end{thm}

\begin{proof}[Sketch of proof]
	WLOG assume $B$ is smooth and of finite type over $k$.
	Let $\tilde{\Sigma}_\ebf(\Ec) = \Quot^{r_1,d_1}(B, \Ec)$, and let $\pi: \tilde{\Sigma} \to \ol{\Sigma}$ be the natural map.
	By normality of $\ol{\Sigma}$, it suffices to show that $\Rbf^{> 0} \pi_* \Oc_{\ol{\Sigma}} = 0$.
	The \emph{Stromme sequence} can be used to embed $\tilde{\Sigma}$ into a certain $F$ such that there exists a flat $\gamma: F \to B$ with fibers $\Quot^{r_1,d_1}(\Oc_{\PP^1}^{\oplus N})$.
	The inclusion $i: \tilde{\Sigma} \to F$ is cut out by a section of a certain vector bundle $\Rc$ and is a local complete intersection.
	We can then reduce to showing $\Rbf^{> m} \gamma_*  \Lambda^m \Rbf^\vee = 0$.
	This can be computed using Schur functors.
\end{proof}

\end{document}
