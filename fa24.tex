\documentclass{amsart}

\usepackage{notes}

\title{Commutative Algebra and Algebraic Geometry Seminar}
\author{Notes by John S.\ Nolan, speakers listed below}
\date{Fall 2024}

\begin{document}

\begin{abstract}
	These are notes from the Berkeley Commutative Algebra and Algebraic Geometry Seminar.
\end{abstract}

\maketitle

\tableofcontents

\section{8/27}

I missed today's talks.
If you have notes and would like to share them, please let me know!

\section{9/3}

I missed today's talks.
If you have notes and would like to share them, please let me know!

\section{9/17a (Peter Haine) -- Reconstructing Schemes from their \'Etale Topoi}

This is based on joint work with M.\ Carlson and S.\ Wolf centered around a conjecture in Grothendieck's anabelian letter to Faltings.

\subsection{Grothendieck's conjecture}

For a scheme $X$, let $X_\et$ be the \'etale topos of $X$ (i.e.\ $\Sh(\Et_X, \Set)$).
We would like to understand when $X$ can be recovered from $X_\et$.
One motivation is to try to understand the following theorem:

\begin{thm}[Neukirch-Uchida 1969, Pop 1994]
	If $K$ and $L$ are infinite fields of the same characteristic that are finitely generated over their prime fields, then there is a bijection between:
	\begin{itemize}
		\item Isomorphisms $K \xrightarrow{\sim} L$
		\item Equivalences of categories $(\Spec L)_\et \xrightarrow{\sim} (\Spec K)_\et$ up to conjugacy.
	\end{itemize}
\end{thm}

Classical statements of this involve the absolute Galois groups, though the statement about \'etale topoi is equivalent.
It is necessary to assume the fields are infinite: for finite fields, the absolute Galois groups are always $\hat{\ZZ}$.

We'll work over a field $k$ for simplicity.
There are four main issues that arise when trying to reconstruct a scheme from its \'etale topos:
\begin{enumerate}
	\item If $L \supset k$ is an extension of separably closed fields, then $(\Spec L)_\et \simeq (\Spec k)_\et$.
		Thus we must restrict to finite type schemes.
	\item If $k$ is an algebraically closed field of characteristic zero and $X$ and $Y$ are smooth proper curves over $k$, then $X_\et \simeq Y_\et$ iff $g(X) = g(Y)$.
		Thus we must restrict to ``small'' fields $k$.
		Grothendieck suggests that we fix this by only considering $k$ which are finitely generated over their prime field.
	\item If $f: X \to Y$ is a universal homeomorphism, then $X \times_Y (-): \Et_Y \to \Et_X$ is an equivalence of categories.
		(Examples include the normalization of the cuspidal cubic, the absolute Frobenius are both universal, and any nil-immersion.)
		Thus we must invert universal homeomorphisms.
	\item A fourth subtle point involving constructibility.
\end{enumerate}

Let's explain this fourth point.
We must start with a small amount of topos theory.

\begin{dfn}
	Given topoi $\Xc$ and $\Yc$, a \emph{geometric morphism} is a functor $f_*: \Xc \to \Yc$ with a right adjoint $f^*: \Yc \to \Xc$ such that $f^*$ preserves geometric morphisms.
\end{dfn}

Every topos $\Xc$ has an associated topological space $\abs{\Xc}$, and $\abs{-}$ is functorial in geometric morphisms.
For a scheme $X$, we have $\abs{X_\et} = \abs{X}$.
Furthermore, if $T$ is a sober topological space, then $\abs{\Sh(T)} \simeq T$.

Knowing this, we can now state our final condition.
If $f: X \to Y$ is a morphism of schemes locally of finite type over a field, then $f$ must send closed points to closed points.
This is not true for general geometric morphisms, so we must require it as an extra condition.

\begin{dfn}
	A geometric morphism $f_*: \Xc \to \Yc$ is \emph{pinned} if $\abs{f_*}: \abs{\Xc} \to \abs{\Yc}$ sends closed points to closed points.
\end{dfn}

One last thing: in the Neukirch-Uchida theorem, we quotiented out by conjugation.
We'd like to know that this doesn't really affect anything (i.e.\ that we shouldn't really be doing something ``stacky'').
Thankfully, this is true:

\begin{prop}
	Let $k$ be a field, and let $X$ and $Y$ be finite type $k$-schemes.
	Then the groupoid $\Hom_k^{\mathrm{pin}}(X, Y)$ of pinned geometric morphisms $X_\et \to Y_\et$ over $(\Spec k)_\et$ is equivalent to a set.
\end{prop}

Now we can state Grothendieck's conjecture:

\begin{conj}[Grothendieck 1983]
	If $k$ is a finitely generated field, then taking \'etale topoi gives an equivalence between:
	\begin{itemize}
		\item $\Sch_k^{ft}[UH\inv]$, the category of finite type $k$-schemes with universal homeomorphisms inverted.
		\item The category of topoi over $(\Spec k)_\et$ and pinned geometric morphisms.
	\end{itemize}
\end{conj}

\begin{thm}[CHW]
	The conjecture is true if $k$ is infinite.
\end{thm}

At present, there is not a complete characterization of the image of $(-)_\et$.

\subsection{Inverting universal homeomorphisms}

We ran into a few issues with universal homeomorphisms.
One was the existence of resolutions of cuspidal cubics.
Another was the existence of the absolute Frobenius.

\begin{dfn}
	A ring $A$ is:
	\begin{enumerate}
		\item \emph{seminormal} if, whenever $x^2 = y^3$, there exists $a \in A$ such that $x = a^2$ and $y = a^2$.
		\item \emph{absolutely weakly normal} (or \emph{awn}) if $A$ is seminormal and, for all primes $\ell$ and equations $\ell^\ell x = y^\ell$, there exists $a \in A$ such that $x = a^\ell$ and $y = \ell a$.
	\end{enumerate}
\end{dfn}

These properties can be extended to general schemes (where we require that they hold affine-locally).

\begin{thm}[Traverso, Swan]
	A ring $A$ is seminormal if and only if the inclusion $A \hookrightarrow A[t]$ induces an isomorphism $\Pic A \xrightarrow{\sim} \Pic A[t]$.
\end{thm}

\begin{thm}
	\begin{enumerate}
		\item The inclusion $\Sch^\awn \hookrightarrow \Sch$ admits a right adjoint $(-)^\awn$ (known as \emph{absolute weak normalization}).
			Moreover, $(-)^\awn$ induces $\Sch[UH\inv] \xrightarrow{\sim} \Sch^\awn$.
		\item The inclusion $\Sch^\sn \hookrightarrow \Sch$ admits a right adjoint $(-)^\sn$ (known as \emph{seminormalization}).
			This identifies $\Sch^\sn$ with the category obtained from $\Sch$ by inverting universal homeomorphisms that induce isomorphisms on residue fields.
		\item A $\QQ$-scheme is awn if and only if it is seminormal.
		\item A $\FF_p$-scheme is awn if and only if it is perfect.
	\end{enumerate}
\end{thm}

In general, seminormalization / absolute weak normalization leaves the land of finite type schemes.
We must allow for this!

\begin{dfn}
	A $k$-scheme $X$ is topologically of finite type if $X \to \Spec k$ factors as $X \to X' \to \Spec k$ where $X \to X'$ is a universal homeomorphism and $X' \to \Spec k$ is of finite type.
\end{dfn}

Grothendieck's conjecture is then equivalent to

\begin{conj}
	For $X$ and $Y$ topologically of finite type over $k$ with $X$ awn, then
	\[
		\Hom_k(X, Y) \simeq \Hom_k^\pin(X_\et, Y_\et).
	\]
\end{conj}

How do we prove this?
Since we can reconstruct the underlying space from the \'etale topos, it suffices to reconstruct the structure sheaf.
This can be rephrased as constructing the sets of morphisms to $\AA^1$.
Using the fact that both sides satisfy $h$-descent together with the theory of alterations, we reduce to the case where $X = (X')^\awn$ for $X'$ regular and finite type.
From here, we can reduce to the case where $Y = \GG_m$.
This lets us understand things cohomologically!
We end up asking questions about Picard groups.

\begin{thm}[Guralnik-Jaffe-Roskind-Wiegland]
	If $k$ is a finitely generated field and $X$ is normal and of finite type over $k$, then $\Pic X$ is finitely generated.
\end{thm}

This is not true for seminormal schemes in general.

\begin{ex}
	If $X \subset \PP^2_k$ is a nodal cubic, then $\Pic X \simeq k^\times \oplus \ZZ$.
\end{ex}

\begin{qn}
	For $X$ a seminormal scheme of finite type over a finitely generated field of characteristic zero:
	\begin{enumerate}
		\item Does $\Pic(X)$ have any nontrivial infinitely divisible elements?
		\item Same as (1) but for torsion elements.
		\item Is the Tate module $T(\Pic(X)) = \lim_n \Pic(X)[n]$ zero?
	\end{enumerate}
\end{qn}

\section{9/17b (Hannah Larson) -- Chow Rings of Moduli Spaces}

\subsection{Chow groups and Chow rings}

Let $X$ be a smooth variety.

\begin{dfn}
	The \emph{Chow group} in codimension $i$ of $X$ is the group of $\ZZ$-linear combinations of codimension $i$ irreducible subvarieties of $X$ modulo rational equivalence.
	Here \emph{rational equivalence} is the equivalence relation generated by $Y_1 \simeq Y_2$ if there exists a family $Y \subset X \times \PP^1$, flat over $\PP^1$, such that $Y_1$ and $Y_2$ are fibers of this family over two points in $\PP^1$.
	We write $[Y]$ for the equivalence class of the subvariety $Y$: this can be defined even for reducible subvarieties by $[Y \cup Z] = [Y] + [Z]$.
\end{dfn}

\begin{dfn}
	The \emph{Chow ring} $A^*(X) = \oplus_i A^i(X)$ comes with an intersection product
	\[
		A^i(X) \times A^j(X) \to A^{i+j}(X)
	\]
	defined so that, if $X$ and $Y$ meet transversally, then $[Y] \cdot [Z] = [Y \cap Z]$.
\end{dfn}

\subsection{Line bundles and $A^*(\PP^n)$}

Let $\Lc$ be a line bundle on $X$, and let $\sigma$ be a rational section of $\Lc$.

\begin{dfn}
	Given $\sigma$, we can define the \emph{divisor of zeroes} $D_0$ and the \emph{divisor of poles} $D_\infty$.
	The difference $c_1(\Lc) := [D_0] - [D_\infty] \in A^1(X)$ does not depend on the choice of $\sigma$.
	We call this the \emph{first Chern class} of $\Lc$.
\end{dfn}

\begin{ex}
	Let $X = \PP^2$ and $\Lc = \Oc_{\PP^2}(1)$.
	Taking $\sigma = x_0$, we obtain $D_0 = V(x_0)$ and $D_\infty = \varnothing$, so $c_1(\Lc) = [V(x_0)]$.
	An alternate choice would be to take $\sigma' = x_1$, which would give $[V(x_1)]$.
	Well-definedness of $c_1(\Lc)$ forces $[V(x_0)] = [V(x_1)]$.

	We can see this directly by considering the graph of the rational map $X \to \PP^1$, $[x_0 : x_1 : x_2] \mapsto [x_0 : x_1]$.
	Taking the closure of the graph gives a rational equivalence between $V(x_0)$ and $V(x_1)$.
	Note that this graph closure is nothing but the blowup of $\PP^1$ at $[1 : 0 : 0]$ -- this observation is more generally.
\end{ex}

The class $\zeta = c_1(\Oc(1)) \in A^1(\PP^n)$ is the first example of a ``tautological class.''
In fact, $A^*(\PP^n) \cong \ZZ[\zeta] / (\zeta^{n+1})$, and $\zeta^i$ is the class of a codimension $i$ hyperplane.
In this case, the ``obvious'' classes generate the Chow ring, and the ``obvious'' relations are all of the relations.

We can obtain B\'ezout's theorem from this result rather easily.

\begin{thm}[B\'ezout]
	If $C$ and $C'$ are curves in $\PP^2$ of degrees $d$ and $d'$ which meet transversally, then $[C] \cdot [C'] = d d' [\pt]$.
\end{thm}

\subsection{More properties of Chow rings}

Chow rings can be defined in the non-smooth case.

\begin{thm}[Excision]
	If $Z \subset X$ is a closed subvariety of codimension $c$ and $U = X \setminus Z$, then there is an exact sequence
	\[
		\begin{tikzcd}
			A^{*-c}(Z) \rar & A^*(X) \rar & A^*(U) \rar & 0.
		\end{tikzcd}
	\]
\end{thm}

We can also pushforward cycles along proper maps.
If $f: X \to Y$ is proper, then $f_*: A^i(X) \to A^{i + \dim Y - \dim X}(Y)$ is defined by
\[
	f_* [Z] = \begin{cases}
		\deg f|_Z \cdot [f(Z)] & \textrm{$f|_Z$ finite} \\
		0 & \textrm{otherwise}
	\end{cases}
\]

\subsection{Moduli spaces of curves}

How can we extend things to moduli spaces of curves?
Let $\Mc$ be the moduli space of curves, and let $f: \Cc \to \Mc$ be the universal curve.
Consider the relative dualizing sheaf $\omega_f$ (so $\omega_f|_p = (T_p C)^\vee$).
We obtain $c_1(\omega_f) \in A^1(\Cc)$ and $f_* c_1(\omega_f) \in A^0(\Mc_g)$.
This gives a ``tautological class'' on $\Mc$.

In fact, we can work a bit more generally: let
\[
	\kappa_i := f_* c_1(\omega_f)^{i+1} \in A^i(\Mc_g).
\]
These are all ``tautological classes.''

\begin{dfn}
	The subring of $A^*(\Mc_g)$ generated by the $\kappa_i$ is the \emph{tautological ring}.
	We write $R^*(\Mc_g)$ for this ring.
\end{dfn}

From now on, we will work rationally, replacing $A^*$ by $A^* \otimes \QQ$.

Important questions:
\begin{itemize}
	\item Does $R^*(\Mc_g) = A^*(\Mc_g)$?
	\item What are the relations among the $\kappa_i$?
		What is $R^*(\Mc_g)$?
\end{itemize}

It is known (by work of several authors) that $R^*(\Mc_g) = A^*(\Mc_g)$ for $g \leq 9$.
However, it is known that $R^*(\Mc_g) \neq A^*(\Mc_g)$ for $g = 12$ or $g \geq 16$.
A heuristic explanation for this is that the lower genus moduli spaces are ``more rational'' than the higher genus moduli spaces.

The non-tautological class in $\Mc_{12}$ is the bielliptic locus $[B_g]$, i.e.\ the locus of curves which admit a degree 2 map to an elliptic curve.
One can show that $[\ol{B}_g]$ is not tautological in $A^*(\ol{\Mc}_g)$ for $g \geq 12$.
The non-tautological classes in $\Mc_{16}$ are constructed as similar ``Hurwitz loci.''

It is not known whether $A^*(\Mc_g)$ is finitely generated in general.
The speaker expects the answer is ``no'' in high genus.

\section{9/24 (Christian Gaetz) -- Combinatorics of Singularities of Schubert Varieties and Torus Orbit Closures Therein}

Let $G$ be a semisimple connected algebraic group over $\CC$.
Most of the setup will work for arbitrary type, but the results will be for groups with simply-laced / ADE Dynkin diagram.

\subsection{Definitions and basic theory}

Let $B$ be a Borel subgroup of $G$, i.e.\ a maximal connected closed solvable subgroup of $G$.
We write $T \subset B$ for the maximal torus and $W = N_G(T) / T$ for the Weyl group.
For the matrix groups, we may take $B$ to consist of the upper triangular matrices in $G$ and $T$ to consist of the diagonal matrices in $G$.

The Weyl group indexes the \emph{Bruhat decomposition} $G = \sqcup_{w \in W} B w B$.
In the flag variety $G / B$, the subspaces $B w B / B \cong \CC^{\ell(w)}$ are the \emph{Schubert cells}.
The closures $X_w = \ol{B w B / B}$ are \emph{Schubert varieties}.

These are classical, hard, and very useful.
For example, they can be used to understand the cohomology of flag varieties: this is the subject of \emph{Schubert calculus}.
We can also relate these to problems in Grassmannians by projecting from flag varieties to Grassmannians.
Schubert calculus in Grassmannians is well-understood (via e.g.\ Littlewood-Richardson rules), but Schubert calculus in Grassmannians is still an active area of research.
We won't focus on Schubert calculus in this lecture.

Note that $X_w = \sqcup_{v \leq w} B w B / B$, where $\leq$ denotes the Bruhat order.

\begin{ex}
	For $G = \SL_n$, we have $W \cong S_n$.
	We can explicitly describe the Bruhat order on $S_3$: writing permutations in one-line notation, we say that $v \leq w$ if the one-line notation for $v$ has more numbers located in the usual order than $w$.
\end{ex}

\subsection{Kazhdan-Lusztig polynomials}

The \emph{Kazhdan-Lusztig polynomials} are $P_{vw}(q) \in \NN[q]$ which admit the following descriptions:
\begin{itemize}
	\item Generating functions of intersection cohomology: $P_{vw}(q) = \sum_i q^i \dim IH^{2i}_v(X_w)$ (where $IH_v$ measures singularities at $v$).
		In the simply laced case, $P_{vw}(q) = 1$ if $X_w$ is smooth at $v$.
		In the non-simply laced case, ``smooth'' is replaced by ``rationally smooth.''
	\item Generating functions of Lie algebra $\Ext$ groups: $\sum_i q^i \Ext^{\ell(w) - \ell(v) - i}_{\gfr}(M_v, L_w)$ (where $M_v$ is a Verma module and $L_w$ is an irrep).
\end{itemize}
Proving the equivalence between these started the field of modern geometric representation theory.
The equivalence is a deep theorem.

We have $\deg P_{vw} \leq \frac{1}{2} (\ell(w) - \ell(v) - 1)$.

The Kazhdan-Lusztig polynomials can be computed via a complicated recurrence relation.
This recurrence relation contains many signs, and we don't know how to prove the nonnegativity of the polynomials from this recurrence relation.

It's impossible to control the behavior of Kazhdan-Lusztig polynomials in general (cf.\ a theorem of Polo ?? saying that every polynomial with constant term $1$ and $\NN$ coefficients arises as a Kazhdan-Lusztig polynomial).
As a result, we'll restrict ourselves to special cases.

\subsection{A conjecture of Billey and Postnikov}

\begin{conj}[Billey-Postnikov]
	Suppose $X_w$ is singular, and write $P_{ew}(q) = 1 + c q^{h(w)} + \dots$.
	If $G$ is simply laced of rank $r$, then $h(w) \leq r - 2$.\footnote{The original statement is a bit weaker.}
\end{conj}

This conjecture is somewhat surprising, as in general we only have $\deg P_{ew} \leq O(r^2)$.
The conjecture forces the first nonzero term that appears to have rank growing linearly in $\rk G$.

\begin{thm}[Bj\"orner-Ekedahl]
	With the above notation, $h(w)$ is the minimal $i$ such that $h^{2i}(X_w) \neq h^{2(\ell(w) - i)}(X_w)$, i.e.\ the first dimension in which Poincar\'e duality fails.
\end{thm}

We can understand this using ``patterns.''

\begin{dfn}
	Say a permutation $w \in S_n$ contains $\sigma \in S_k$ as a \emph{pattern} if there exist $1 \leq i_1 < \dots < i_k < n$ such that $w(i_a) < w(i_b)$ if and only if $\sigma(a) < \sigma(b)$.
	Otherwise we say $w$ \emph{avoids} $\sigma$.
\end{dfn}

\begin{ex}
	The permutation $w = 45312 \in S_5$ contains $\sigma = 3412 \in S_4$ as a pattern (via the subpermutation $4512$).
\end{ex}

\begin{thm}[Lakshmibai-Sandhya]
	The Schubert variety $X_w$ is smooth if and only if $w$ avoids $3412$ and $4231$.
\end{thm}

Avoiding patterns is hard to do in high rank!
In particular, $X_w$ is almost always singular.

\subsection{The theorem}

\begin{thm}[Gaetz-Gao]
	Suppose $X_w$ is singular.
	Then
	\[
		h(w) = \begin{cases}
			1 & \textrm{$w$ contains $4231$} \\
			\textrm{minHeight}(w) & \textrm{otherwise}
		\end{cases}
	\]
	where
	\[
		\mathrm{minHeight}(w) = \min \{ w(i_1) - w(i_4) \,|\, \textrm{$w$ has $3412$ in positions $i_1, i_2, i_3, i_4$}\}.
	\]
\end{thm}

For any $w$ containing $3412$, we have $\mathrm{minHeight}(w) \leq n - 3 = (n - 1) - 2$.
The Billey-Postnikov conjecture follows from this.

The proof is easier in the case where $w$ contains $4231$.
Otherwise, $w$ avoids $4231$, and one considers projections $G/B \to G/P_J$.
The images of $X_w$ give Schubert varieties $X^J_{W^J} \subset G/P_j$.
The projection maps $\pi: X_w \to X^J_{W^J}$ can be rather nasty, but by studying $w$ carefully, one finds $J$ such that $\pi$ is a fiber bundle with fiber $X_{W_J}$ (i.e.\ a smaller Schubert variety!).
This allows one to understand Schubert varieties inductively.

\section{10/1a (Daigo Ito) -- A New Proof of the Bondal-Orlov Reconstruction Theorem}

This is based on joint work with Hiroki Matsui.

\subsection{Background}

Let $X$ be a projective variety over $\CC$ (though this should mostly work over any algebraically closed field).
We can construct a triangulated category of perfect complexes $\Perf\, X$ consisting of bounded complexes of vector bundles (after inverting quasi-isomorphisms).

\begin{qn}
	How much information about $X$ is contained in $\Perf\, X$?
	In particular, in which cases can we reconstruct $X$ fully from $\Perf\, X$?
\end{qn}

\begin{thm}[Bondal-Orlov, Ballard]
	Let $X$ be Gorenstein, so the dualizing complex $\omega_X$ is a line bundle.
	If $\omega_X$ is ample or anti-ample, then:
	\begin{enumerate}
		\item We can reconstruct $X$ from the triangulated category $\Perf\, X$.
		\item If $\Perf\, X \simeq \Perf\, Y$ for $Y$ projective and Gorenstein, then $X \cong Y$.\footnote{This is needed to ensure that our reconstruction gives the correct result!
			Note that we do not assume $Y$ is Fano / anti-Fano.}
	\end{enumerate}
\end{thm}

The Fano / anti-Fano assumptions here are rather strong.
Another method of reconstructing $X$ is as follows.

\subsection{Balmer spectra}

\begin{thm}[Balmer]
	Let $X$ be a variety (more generally, a qcqs scheme).
	Then we can reconstruct $X$ from $(\Perf\, X, \otimes_{\Oc_X}^L)$.
	
	More precisely, for any tensor-triangulated (tt) category $(\Tc, \otimes)$, we can construct a ringed space $\Spec_\otimes \Tc$.
	Taking $(\Tc, \otimes) \cong (\Perf\, X, \otimes_X)$, we obtain $\Spec_\otimes \Tc \cong X$.
\end{thm}

It is natural to ask if we can perform Balmer's construction without having the tensor structure when $\omega_X$ is anti-ample.


\begin{dfn}[Balmer]
	Let $(\Tc, \otimes)$ be a tt-category.
	The \emph{Balmer spectrum} $\Spec_\otimes \Tc$ is defined as follows.
	As a set, $\Spec_\otimes \Tc$ consists of \emph{prime thick $\otimes$-ideals} $\Pc \subset \Tc$, meaning that:
	\begin{enumerate}
		\item (Thick subcategory) $\Pc$ is a full triangulated subcategory of $\Tc$, such that if $X \oplus Y \in \Pc$, then $X \in \Pc$ or $Y \in \Pc$.
		\item ($\otimes$-ideal) If $X \in \Pc$ and $Y \in \Tc$, then $X \otimes Y \in \Pc$.
		\item (Prime) $\Pc$ is a proper $\otimes$-ideal such that, if $X \otimes Y \in \Pc$, then $X \in \Pc$ or $Y \in \Pc$.
	\end{enumerate}
	We can equip this with a natural ringed space structure.
\end{dfn}

\begin{ex}
	If $X$ is a qcqs scheme, then $X \cong \Spec_{\otimes_{\Oc_X}} \Perf\, X$, where a (not necessarily closed) point $x \in X$ corresponds to the prime
	\[
		S_X(x) = \big\{ \Fc \in \Perf\, X \big| \Fc_x \simeq 0 \textrm{ in } \Perf(\Oc_{X,x}) \big\}
	\]
\end{ex}

\subsection{Matsui spectra}

\begin{dfn}[Matsui]
	Let $\Tc$ be a triangulated category.
	The \emph{Matsui spectrum} $\Spec_\Delta \Tc$ is defined as the set of prime thick subcategories $\Pc$, meaning that:
	\begin{enumerate}
		\item $\Pc$ is a thick subcategory of $\Tc$.
		\item The collection $\{ \Qc \supsetneq \Pc \,|\, \Qc \textrm{ thick}\}$ has a unique smallest element.
	\end{enumerate}
\end{dfn}

\begin{ex}
	If $X$ is a curve and $x \in X$ is a closed point, then $S_X(x) = \langle\kappa(y) \,|\, y \neq x\rangle$ is prime.
	The smallest thick subcategory containing $S_X(x)$ is obtained by adjoining $\kappa(x)$ to $S_X(x)$.
\end{ex}

\begin{prop}[Matsui]
	If $X$ is a noetherian scheme, then $\Spec_{\otimes_{\Oc_X}} \Perf\, X \subset \Spec_\Delta \Perf\, X$ as sets.
\end{prop}

\begin{rmk}
	It is not known yet whether this extends to arbitrary qcqs schemes.
\end{rmk}

\begin{thm}[Ito-Matsui]
	If $X$ is a quasiprojective scheme, then $(\Spec_{\otimes_{\Oc_X}} X)_{\mathrm{red}} \subset \Spec_\Delta \Perf\, X$ is an open immersion of ringed spaces.
\end{thm}

\begin{rmk}
	It is not known yet whether we can attach a natural structure sheaf to $\Spec_\Delta \Perf\, X$ that ``sees the nilpotents of $X$.''
\end{rmk}

\subsection{Proof of Bondal-Orlov}

Assume $\omega_X$ is (anti)ample.
By the theorem on open immersions above, it suffices to specify the correct open subspace of the Matsui spectrum.

Let $\SS(-) = - \otimes \omega_X[\dim X]$ be the Serre functor of $\Perf\, X$.
This gives an automorphism of $\Perf\, X$.

\begin{lem}
	The Serre functor can be uniquely determined by the triangulated category structure of $\Perf\, X$.
\end{lem}

With this lemma, we can define the \emph{Serre-invariant locus} $\Spec^{\mathrm{Ser}} \Perf\, X \subset \Spec_\Delta \Perf\, X$ as 
\[
	\Spec^{\mathrm{Ser}} \Perf\, X = \big\{ \Pc \in \Spec_\Delta \Perf\, X \,\big|\, \SS(\Pc) = \Pc \big\}
\]
We claim $\Spec_{\otimes_{\Oc_X}} \Perf\, X = \Spec^{\mathrm{Ser}} \Perf\, X$.

\begin{ex}
	For $X = \PP^1$, we have
	\[
		\Spec_\Delta \Perf\, \PP^1 = \Spec_{\otimes_{\Oc_{\PP^1}}} \sqcup \coprod_{i \in \ZZ} \angles{\Oc_{\PP^1}(i)},
	\]
	and $\SS$ fixes $\Spec_{\otimes_{\Oc_{\PP^1}}}$ while acting freely on $\coprod_{i \in \ZZ} \angles{\Oc_{\PP^1}(i)}$.
	Thus the claim holds in this case.
\end{ex}

In general, it is clear that $\Spec^{\mathrm{Ser}} \Perf\, X \subset \Spec_{\otimes_{\Oc_X}} \Perf\, X$.
For the reverse inclusion, suppose $\SS(\Pc) = \Pc$.
Then, for all $n \in \ZZ$, we have $\Pc \otimes \omega_X^{\otimes n} = \Pc$.

\begin{thm}[Orlov]
	If $\omega_X$ is (anti)ample, then $\langle \omega_X^{\otimes n} \,|\, n \in \ZZ\rangle = \Perf\, X$.
\end{thm}

It follows that, for all $\Fc \in \Perf\, X$, we have $\Fc \otimes \Pc \subset \Pc$, i.e.\ $\Pc$ is a $\otimes$-ideal.

\begin{thm}[Matsui]
	A $\otimes$-ideal is a prime $\otimes$-ideal if and only if it is a prime thick subcategory.
\end{thm}

Thus $\Pc$ is a prime $\otimes_{\Oc_X}$-ideal of $\Perf\, X$, i.e.\ a point of $\Spec_{\otimes_X} \Perf\, X$.
This concludes the reconstruction of $X$ from $\Perf\, X$.

To see that $\Perf\, X \simeq \Perf\, Y$ implies $X \simeq Y$, note that, because the Serre functor commutes with any equivalence, we must have 
\[
	X \cong \Spec^{\mathrm{Ser}} \Perf\, X \cong \Spec^{\mathrm{Ser}} \Perf\, Y.
\]
We know that $Y$ embeds as an open subspace of $\Spec^{\mathrm{Ser}} \Perf\, Y$.
Thus $Y$ is a closed (because proper) and open subspace of $X$.
But $X$ is connected, so $X \cong Y$.

\subsection{More applications}

Using these methods, we can also recover a theorem of Favero.

\begin{thm}[Favero]
	Let $X$ and $Y$ be projective varieties.
	Suppose there is an equivalence $\Phi: \Perf\, X \xrightarrow{\sim} \Perf\, Y$, an (anti)ample line bundle $\Ac$ on $X$, and a line bundle $\Lc$ on $Y$, such that $\Phi(- \otimes \Ac) \simeq \Phi(-) \otimes \Lc$.
	Then $X \cong Y$.
\end{thm}

\begin{proof}
	By the above, we have $X \subset (\Spec_\Delta \Perf\, X)^{- \otimes \Ac} = (\Spec_\Delta \Perf\, Y)^{- \otimes \Lc}$.
	The same argument as above gives $X \simeq Y$.
\end{proof}

The above proof is much shorter than Favero's.

\begin{rmk}
	We do not yet know how to reconstruct the tensor product $\otimes_{\Oc_X}$ directly from the Serre functor.
\end{rmk}

\section{10/1b (Noah Olander) -- Fully Faithful Functors and Dimension}

We'll write $X$ and $Y$ for smooth projective varieties over some field $k$.
We write $\Db(X) := \Db(\Coh(X)) \simeq \Perf\, X$, viewed as a $k$-linear triangulated category.
All functors considered will be exact and $k$-linear.

\subsection{The main theorem}

Our goal is to prove the following:

\begin{thm}[Theorem 1]
	If there exists a fully faithful functor $F: \Db(X) \to \Db(Y)$, then $\dim X \leq \dim Y$.
\end{thm}

Let's give some examples of fully faithful functors to explain what we mean:
\begin{enumerate}
	\item If $f: Y \to X$ satisfies $Rf_* \Oc_Y = \Oc_X$, then $Lf^*: \Db(Y) \to \Db(X)$ is fully faithful.
		This applies for $Y$ a projective bundle over $X$, a blowup of $X$ with smooth center, and many other cases.
		In this case, $f$ must be surjective, proving the theorem in this case.
	\item (Kuznetsov) There exist fully faithful functors from derived categories of K3 surfaces to derived categories of cubic fourfolds.
\end{enumerate}
In these and other examples, the theorem is obvious.
But the result in general wasn't known before!

We will prove the theorem by proving a weaker version of Orlov's conjecture, stated as follows.
Let $\Rdim$ denote the Rouquier dimension of a triangulated category, to be defined later.

\begin{conj}[Orlov]
	For a smooth projective variety $X$, we have $\Rdim \Db(X) = \dim X$.
\end{conj}

In this generality, the conjecture implies Theorem 1.

\subsection{Rouquier dimension}

Let $\Tc$ be a triangulated category and $S \subset \Tc$.
Let $\angles{S}_{d+1} \subset \Tc$ consist of objects in $\Tc$ built by taking direct sums, shifts, passing to direct summands, and \emph{taking at most $d$ cones}.

\begin{dfn}
	The \emph{Rouquier dimension} $\Rdim \Tc$ is the smallest integer $d$ such that there exists a single object $G \in \Tc$ with $\Tc = \angles{G}_{d+1}$.
\end{dfn}

We will make use of a variant notion.

\begin{dfn}
	The \emph{countable Rouquier dimension} $\CRdim \Tc$ is the smallest integer $d$ such that there exists a countable subset $S \subset \ob \Tc$ with $\angles{S}_{d+1}$.
\end{dfn}

\begin{ex}
	Let $R$ be a Dedekind domain and $\Tc$ the bounded derived category of finitely generated $R$-modules.
	Then $\Rdim \Tc = 1$, with $\Tc = \angles{R}_2$.
	To see this, let $K \in \Tc$.
	Then we may write $K = \oplus_{i \in \ZZ} H^i(K)[-i]$, so WLOG $K$ is a finitely generated $R$-module concentrated in degree zero.
	Then $K$ admits a resolution
	\[
		\SES{P}{R^{\oplus d}}{K}
	\]
	where $P$ is projective.
	As $P$ and $R^{\oplus d}$ both lie in $\angles{K}_1$ (since $P$ can be obtained as a direct summand of a free module), we have $K \in \angles{R}_2$.
	It follows that $\Rdim \Tc \leq 1$.
	The converse is left as an exercise.

	Note that if $R$ is countable, then $\CRdim \Tc = 0$, so $\CRdim$ and $\Rdim$ don't agree.
\end{ex}

\begin{ex}
	Let $X$ be a smooth projective curve of genus $g \geq 1$.
	Let $\Lc$ be a line bundle on $X$ of degree $\geq 8g$.
	Orlov showed $\Db(X) = \angles{\Lc\inv \oplus \Oc \oplus \Lc \oplus \Lc^{\otimes 2}}_2$.
	Thus $\Rdim X \leq 1$, and equality holds by the following.
\end{ex}

\begin{lem}[Rouquier]
	We have $\Rdim \Db(X) \geq \dim X$.
\end{lem}

The same argument also shows $\CRdim \Db(X) \geq \dim X$ if the ground field $k$ is uncountable.

\begin{lem}
	Let $F: \Tc \to \Tc'$ be a fully faithful exact functor with an exact right adjoint.
	Then $\Rdim \Tc \leq \Rdim \Tc'$.
\end{lem}

\begin{proof}
	If $R$ is the right adjoint, then $R \circ F = \id$, so $R$ is essentially surjective.
	If $G$ is an optimal generator for $\Tc$, then $R(G)$ generates $\Rdim \Tc$ in (at most) the same number of steps.
\end{proof}

This lemma (combined with a result of Bondal and van den Bergh on existence of adjoints) implies that Theorem 1 holds if Orlov's conjecture holds.
The lemma also holds with $\Rdim$ replaced by $\CRdim$.

\subsection{A weaker version of Orlov's conjecture}

Theorem 1 also follows from:\footnote{Some dexterity is needed to reduce the general case to the case where $k$ is uncountable, but this is not too difficult.}

\begin{thm}[Theorem 2]
	Let the ground field $k$ be uncountable.
	Then $\CRdim \Db(X) \leq \dim X$.
	More precisely, $\Db(X) = \angles{\{\Oc_X(i)\}_{i \in \ZZ}}_{\dim X + 1}$.
\end{thm}

\begin{thm}[Theorem 3]
	Given maps $K_0 \to K_1 \to \dots \to K_{\dim X + 1}$ in $\Db(X)$ such that $H^n(K_i) \to H^n(K_{i+1})$ is zero for all $i$ and $n$, the composite map $K_0 \to K_{\dim X + 1}$ is zero.
\end{thm}

Maps $K_i \to K_{i+1}$ which are nonzero despite all maps $H^n(K_i) \to H^n(K_{i+1})$ being zero are called ``ghosts'' in topology.
They are relatively easy to find.

\begin{ex}
	Let $\xi \in H^1(\PP, \Oc(-2))$ be nonzero.
	Then $\xi$ gives a nonzero element of $\Hom(\Oc, \Oc(-2)[1])$, despite the fact that $\Oc$ and $\Oc(-2)[1]$ are concentrated in different cohomological degrees.
\end{ex}

\begin{proof}[Proof of Theorem 3]
	Consider the spectral sequence $E_1^{p,q} = \prod_n \Ext^{2p+q}(H^n(K), H^{n-p}(L))$ which converges to $\Ext^{p+q}(K, L)$.
	This gives a filtration $F^\bullet$ on $\Hom(K, L)$.
	We can see that $F^{\dim X + 1} = 0$, $F^1$ consists of ghost maps, and $F^r \circ F^s \subset F^{r+s}$.
	The result follows from these properties.
\end{proof}

\begin{proof}[Proof of Theorem 2]
	Let $K_0 = K \in \Db(X)$.
	Choose a finite direct sum $\oplus \Oc_X(n_i)^{\oplus d_i}[e_i]$ together with a map $\phi_0: \oplus \Oc_X(n_i)^{\oplus d_i}[e_i] \to K_0$ which is surjective on cohomology.
	Let $K_1 = \cone \phi_0$, so $K_0 \to K_1$ is a ghost.
	Repeat this inductively to get $K_0 \to \dots \to K_{\dim X + 1}$.
	
	We claim that $\cone(K_0 \to K_j) \in \angles{\{\Oc_X(i)\}_{i \in \ZZ}}_j$.
	For $j = 1$, this is true by construction.
	For $j > 1$, we use the octahedral axiom to construct an exact triangle
	\[
		\cone(K_0 \to K_{j-1}) \to \cone(K_0 \to K_j) \to \cone(K_{j-1} \to K_j) \to
	\]
	and use this to deduce the claim.

	The theorem follows by considering $K_{\dim X + 1} \oplus K_0[1] \simeq \cone(K_0 \xrightarrow{0} K_{\dim X + 1}) \in \angles{\{\Oc_X(i)\}_{i \in \ZZ}}_{\dim X + 1}$.
\end{proof}

\section{10/8 (Hannah Larson) -- The Chow Rings of Moduli Spaces of Pointed Hyperelliptic Curves}

This is related to the speaker's prior talk in this seminar, but knowledge of that talk is not necessary for this.

\subsection{Refresher on Chow rings}

Let $X$ be a smooth variety.
The \emph{Chow ring} $A^*(X)$ is the ring of $\ZZ$-linear combinations of irreducible subvarieties $Y \subset X$, modulo rational equivalence.
Recall that $Y_1$ and $Y_2$ are rationally equivalent (written $Y_1 \sim Y_2$) if there is a subvariety $Z \subset X \times \PP^1$ such that the forgetful map $p: Z \to \PP^1$ is flat and for some $t_1, t_2 \in \PP^1$ we have $p\inv(t_1) = Y_1$ and $p\inv(t_2) = Y_2$.
The Chow ring is graded by codimension.
Addition in the Chow ring corresponds to taking unions, and multiplication corresponds to taking (suitable) intersections.

We will need to know a few computational facts about Chow rings:
\begin{enumerate}
	\item $A^*(\AA^n) = \ZZ$, concentrated in codimension $0$.
		For example, the subvariety $V(x^2 + y^2 - 1) \subset \AA^2$ is rationally equivalent to $\varnothing$ via the family $V((tx^2) + (ty)^2- 1) \subset \AA^2 \times \PP^1_t$.
	\item There is an \emph{excision property}: if $Z \subset X$ is closed of codimension $c$, then there is an exact sequence
		\[
			\begin{tikzcd}
				A^{*-c}(Z) \rar & A^*(X) \rar & A^*(U) \rar & 0
			\end{tikzcd}
		\]
		where $U = X \setminus Z$.
\end{enumerate}

These can be used to compute $A^*(\PP^n) = \ZZ[\zeta] / (\zeta^{n+1})$, using induction and the fact that
\[
	\PP^n = \AA^n \sqcup \PP^{n-1} = \AA^n \sqcup \AA^{n-1} \sqcup \dots \sqcup \pt.
\]
To make this calculation work, one does need to know that the class of a point in $A^*(\PP^n)$ is nonzero, which follows from proper pushforward being well-defined.
We will use similar stratification-based methods today.

\subsection{Moduli of curves}

Let $\Mc_g$ be the (coarse) moduli space of curves.
One way to stratify this by gonality.

\begin{dfn}
	The \emph{gonality} of a curve $C$ is the minimal $k$ such that there exists a degree $k$ map $C \to \PP^1$.
\end{dfn}

For curves of genus $g \neq 0$, the minimum gonality is $2$.
Curves of gonality $2$ are called \emph{hyperelliptic} and determine a closed subvariety $\Hc_g \subset \Mc_g$.
Using the Riemann-Hurwitz formula, we see that if $C$ is a hyperelliptic curve of genus $g$, the ramification divisor $R$ of any degree $2$ map $C \to \PP^1$ satisfies $\deg R = 2g + 2$.
In particular, the coarse moduli space $\Hc_g$ corresponds to the space of collections of $2g + 2$ distinct points on $\PP^1$, all modulo automorphisms of $\PP^1$.\footnote{This is not quite true for the stacks, but we will not concern ourselves with this.}
This can be written as
\[
	(\PP^{2g+2} \setminus \Delta) / \PGL_2
\]
where $\Delta$ is the locus where the points collide.
In particular, we note that the coarse moduli space of $\Hc_g$ is unirational.

We can compute the Chow ring of this space (with rational coefficients) using equivariant intersection theory.
The answer is
\[
	A^*(\Hc_g) \otimes \QQ = \QQ
\]
Since we are using rational coefficients, the coarse moduli space gives the same answer as the stacky computation.

\subsection{Introducing marked points}

Let $\Hc_{g,n}$ be the moduli space of smooth hyperelliptic curves with $n$ distinct marked points.
We have good reason to care about these: the boundary of the compactification $\ol{\Mc}_{g'}$ can be stratified by products of moduli spaces $\Mc_{g,n}$ of curves with marked points, and the $\Hc_{g,n}$ appear in the stratifications of these moduli spaces by gonality.

Let's consider $\Hc_{g,1}$ first.
For a hyperelliptic curve $C$ with $f: C \to \PP^1$, if $p$ is a marked point on $C$, there is an automorphism of $C$ taking $p$ to the other point $\ol{p} \in f\inv(f(p))$.
Thus all we really care about is the image $f(p)$, i.e.\ $\Hc_{g,1}$ has coarse moduli space given by the space of collections of $2g+2$ distinct points on $\PP^1$ together with one distinguished point on $\PP^1$.
As a variety, this is
\[
	((\PP^2 \setminus \Delta) \times \PP^1) / \PGL_2.
\]
The forgetful map $\Hc_{g,1} \to \Hc_g$ is a $\PP^1$-bundle, and we get
\[
	A^*(\Hc_{g,1}) \otimes \QQ = \QQ[\phi] / (\phi^2).
\]

As $n$ grows, the spaces $\Hc_{g,n}$ become more complicated.
This can be made precise:

\begin{thm}[Barros-Mullane, Schwarz]
	For $n \geq 4g + 7$, the space $\Hc_{g,n}$ is of general type.
\end{thm}

In particular, it is not possible to find a dominant map from a rational variety to $\Hc_{g,n}$ (i.e.\ a ``parametrization'') for $n$ large.
Nevertheless, $\Hc_{g,n}$ is easier to understand in low degrees:

\begin{thm}[Casnati]
	For $n \leq 2g + 8$, the space $\Hc_{g,n}$ is rational.
\end{thm}

\begin{thm}[Canning-Larson]
	For $n \leq 2g + 6$, we have
	\[
		A^*(\Hc_{g,n}) \otimes \QQ = \frac{\QQ[\phi_1, \dots, \phi_n]}{(\phi_1, \dots, \phi_n)^2}
	\]
	where the classes $\phi_i$ all lie in codimension $1$.
	Furthermore, $\Hc_{g,n}$ is rational for $n \leq 3g + 6$.
\end{thm}

The classes $\psi_i$ here are tautological classes.
Over $\Hc_{g,n}$, there is a universal hyperelliptic curve $f: \Cc \to \Hc_{g,n}$ with disjoint sections $\sigma_1, \dots, \sigma_n: \Hc_{g,n} \to \Cc$.
One defines $\psi_i = c_1(\sigma_i^* \omega_f)$.
This is useful for understanding $A^*(\Mc_{g,n}) \otimes \QQ$: the only contributions to this coming from $\Hc_{g,n}$ will be tautological!

Canning and Larson also have some similar results for higher gonality.

\section{10/15a (Smita Rajan) -- Kinematic Varieties for Massless Particles}

This is joint work with Svala Sverrisdottir and Bernd Sturmfels.
The goal is to understand scattering amplitudes in theoretical physics.

\subsection{Background}

One can specify a QFT by specifying a Lagrangian $\Lc$.
From this Lagrangian, one can extract Feynman diagrams, which describe the interaction of particles.
``Tree-level'' Feynman diagrams (those without loops) are the easiest to work with and produce the largest physical effects.
To compute scattering amplitudes and other physical quantities from Feynman diagrams, we need to compute Feynman integrals.
These are hard in general, but if we just care about tree-level scattering amplitudes, we can use the ``spinor-helicity formalism'' (which we'll discuss later).

The spinor-helicity formalism is well-understood in $4$ ($=3+1$) dimensions.
However, extending it to higher dimensions turns out to be an interesting problem because of the relations between variables.

\subsection{Particles in $d$-dimensional spacetime}

\begin{dfn}
	By $d$-dimensional spacetime, we mean the real vector space $\RR^d$ with the Lorentzian inner product
	\[
		x \cdot y = -x_1 y_1 + x_2 y_2 + \dots + x_n y_n.
	\]
\end{dfn}

We consider a configuration of $n$ particles in $d$ dimensions.
Each particle has a momentum $p_i = (p_{i,1}, \dots, p_{i,d})$.
For massless particles, we obtain $n$ quadric relations ($p_i^2 = 0$ for $1 \leq i \leq n$).
We also impose momentum conservation, giving $d$ linear relations ($\sum_{i=1}^n p_{ij} = 0$ for $1 \leq j \leq d$).

Algebraically, we focus on the polynomial ring $\CC[p_{ij}]$ in $n d$ variables.\footnote{The complexification occurs here for physical reasons -- certain quantities aren't well-defined if we only work with real numbers.}
Let $I_{d,n}$ be the ideal generated by the aforementioned $n$ quadrics and $d$ linear forms.
Then $V(I_{d,n})$ is irreducible and of the expected dimension ($nd - n - d$).

\begin{thm}
	The ideal $I_{d,n}$ is prime and is a complete intersection if $\max(d, n) \geq 4$.
\end{thm}

\begin{proof}
	First consider the case $d \geq 5$.
	By eliminating the variable $p_n$, we can write $I_{d,n} = J + \angles{f}$, where $J$ is generated by $n - 1$ quadrics and 
	\[
		f = \left(\sum_{i=1}^{n-1} p_i\right) \cdot \left(\sum_{i=1}^{n-1} p_i\right).
	\]
	We can write
	\[
		\frac{\CC[p_1, \dots, p_{n-1}]}{J} \cong \frac{\CC[p_1]}{p_1 \cdot p_1} \otimes \dots \otimes \frac{\CC[p_{n-1}]}{p_{n-1} \cdot p_{n-1}}.
	\]
	A Hartshorne exercise shows that each of the factors has trivial divisor class group, so the ring $\CC[p_1, \dots, p_{n-1}] / J$ is a UFD.
	One can use this to show that the ideal in question is prime.

	For $d = 3$ and $n \geq 4$, we can use Serre's criterion to show that $\CC[p_i] / \angles{p_i \cdot p_i}$ is normal.
	This can be used to show the result.

	The $d = 4$ case was handled in the paper ``Spinor Helicity Varieties.''
\end{proof}

\subsection{The Clifford algebra and spinors}

\begin{dfn}
	Let $V$ be a vector space equipped with a bilinear form $B$.
	The Clifford algebra is
	\[
		\Cl(V) = T(V) / \angles{v \otimes w + w \otimes v - 2B(v, w)}.
	\]
	We will write $\Cl(1, d - 1)$ for the Clifford algebra corresponding to the Lorentzian inner product $\eta$ on $\CC^d$.
	This can be written explicitly as
	\[
		\Cl(1, d-1) = \frac{\CC\angles{\gamma_1, \dots, \gamma_d}}{\angles{\gamma_i \gamma_j + \gamma_j \gamma_i - 2 \eta_{ij}}}
	\]
\end{dfn}

\begin{rmk}
	The associated graded of the Clifford algebra is just the usual exterior algebra.
\end{rmk}

We'd like to construct a matrix representation of $\Cl(1, n-1)$ of dimension $2^k$, where $k = \lfloor d / 2 \rfloor$.
For $d = 2$, we send $\gamma_1$ and $\gamma_2$ to the matrices
\[
	\Gamma_1 = \begin{bmatrix}0 & 1 \\ -1 & 0 \end{bmatrix}, \hspace{2em} \Gamma_2 = \begin{bmatrix}0 & 1 \\ 1 & 0\end{bmatrix}
\]
We can define the representations in dimensions $d > 2$ inductively.

For $d = 2k$, we send $\gamma_i$ for $1 \leq i \leq d-2$ to
\[
	\Gamma_{2k,i} = \Gamma_{k-1,i} \otimes \begin{bmatrix}-1 & 0 \\ 0 & 1\end{bmatrix}.
\]
We can also define matrices for $\gamma_{d-1}$ and $\gamma_d$.
Similar arguments work for odd $d$.

These representations give rise to the \emph{spinor representations} of the Lie algebra $\sofr(1, d-1)$.
Specifically, send the $ij$th generator of $\sofr(1, d-1)$ to $\Sigma_{ij} = \frac{1}{4} [\Gamma_i, \Gamma_j]$.

\subsection{Back to particles}

For each particle, we construct a \emph{momentum space Dirac matrix}
\[
	P_i = -p_{i1} \Gamma_1 + p_{i2} \Gamma_2 + \dots + p_{id} \Gamma_d.
\]

\begin{dfn}
	The \emph{charge conjugation matrix} $C$ is an equivariant map from the spinor representation of $\sofr(1, d-1)$ to its dual satisfying
	\[
		C \Gamma_i = -\Gamma_i^T C.
	\]
\end{dfn}

\begin{ex}
	For $d = 4$, we have
	\[
		C = \begin{bmatrix}
			0 & -i & 0 & 0 \\
			i & 0 & 0 & 0 \\
			0 & 0 & 0 & -i \\
			0 & 0 & i & 0
		\end{bmatrix}
	\]
\end{ex}

We'd like to parametrize the column space of the matrix $P_i$ by the variables 
\[
	z_i = (z_{i,1}, \dots, z_{i,2^{k-2}}, 0, \dots, 0, z_{i,2^{k-1}+1}, \dots, z_{2^k})^T.
\]
We define the \emph{spinor helicity variable} as $\ket{i} = P_i z_i$.
We also let $\bra{i} = (P_i z_i)^T$.

We can use these to define \emph{spinor brackets}:
\begin{itemize}
	\item Order 2: $\angles{ij} = \bra{i} C \ket{j} = z_i^T P_i^T C P_j z_j$
	\item Order 3: $\angles{ijk} = \bra{i} C P_j \ket{k}$
	\item Higher order: defined similarly.
\end{itemize}

\begin{thm}
	These spinor brackets are invariant under the $\sofr(1, d-1)$ action in the spinor representation (via conjugation by the $P$-matrices and left multiplication on the $z$-variables).
\end{thm}

Consider matrices $S$ and $T_j$, where $S_{ij} = \angles{ij}$ and $(T_j)_{ik} = \angles{ijk}$.

\begin{dfn}
	The \emph{kinematic variety} $K_{d,n}^{(2)}$ is the variety of possible matrices $S$ as above.
\end{dfn}

\begin{thm}
	For $d = 3$, the ideal of $K_{3,n}^{(2)}$ is generated by the $4 \times 4$ Pfaffians of a skew-symmetric matrix $n \times n$ matrix, so $K_{3,n}^{(2)} \cong \Gr(2, n)$.
	For $d = 4$ and $d = 5$, we have $K_{4,n}^{(2)} \cong K_{5,n}^{(2)}$, and these varieties are cut out by $6 \times 6$ Pfaffians of a skew-symmetric $n \times n$ matrix.
	In particular, $K_{4,n}^{(2)} \cong K_{5,n}^{(2)}$ is the first secant variety of $\Gr(2, n)$.
\end{thm}

\section{10/22 (Will Fisher) -- Introduction to Hochschild Homology}

\subsection{Introduction and motivation}

Let $A$ be an associative $k$-algebra (where $k$ is unital and commutative).
What is the ``universal'' commutative algebra associated to $A$?

We can interpret this question categorically: what are the adjoints to $\CAlg_k \hookrightarrow \Alg_k$?
One adjoint is given by $A \mapsto A / [A, A]$, but the other doesn't exist.
Trying to take the center doesn't work, as it's not functorial in general.
Let's fix this by considering a more general problem.

Given an $A / k$-bimodule (``$A$-over-$k$ bimodule'') $M$, we'd like to find the universal bimodule where the left and right actions agree.
We recall the definition of $A / k$-bimodules, as this is somewhat more restrictive than the na\"ive notion of bimodule.

\begin{dfn}
	An $A / k$-bimodule is a $k$-module $M$ with the structure of a left and right $A$-module such that
	\begin{enumerate}
		\item $(a \cdot m) \cdot b = a \cdot (m \cdot b)$ for $a, b \in A$ and $m \in M$
		\item $\lambda \cdot m = m \cdot \lambda$ for $\lambda \in k$ and $m \in M$.
	\end{enumerate}
\end{dfn}

There are two obvious options for these ``universal bimodules:''
\begin{enumerate}
	\item $Z(M) = \{ m \in M \,|\, am = ma \forall a \in A \}$
	\item $M / [A, M] = M / \angles{am - ma \,|\, m \in M, a \in A}$
\end{enumerate}

If $M$ is an $A / k$-bimodule such that $a \cdot m = m \cdot a$ for all $a \in A$, then $[A, A]$ annihilates $M$.
In fact, this commutativity condition is equivalent to an $A / [A, A]$-left module structure on $M$.
Thus we have an inclusion $(A / [A, A])\dMod \hookrightarrow A / k\dBimod$.
One can check that $M \mapsto Z(M)$ and $M \mapsto M / [A, M]$ are the left and right adjoints to this inclusion.

\begin{dfn}
	\emph{Hochschild homology} $\HHrm_*(A/k, -)$ is the derived functor of 
	\begin{align*}
		A / k\dBimod &\to (A / [A, A])\dMod \\
		M &\mapsto M / [A, M].
	\end{align*}
\end{dfn}

If $A$ is a \emph{commutative} algebra, then an $A / k$-bimodule $M$ is an $A \otimes_k A$-module via $(a \otimes b) \cdot m = a \cdot m \cdot b$.
A similar construction lets us also view $M$ as a right $A \otimes_k A$-module.
Let $\iota: A\dMod \to (A \otimes_k A)\dMod$ be defined by $(a \otimes b) \cdot m = ab \cdot m$, so we are viewing $M$ as an $A \otimes_k A$-module via the multiplication $m: A \otimes_k A \to A$.
Geometrically, this corresponds to pushforward along the diagonal $\Delta: \Spec A \to \Spec A \times_k \Spec A$.

\begin{cor}
	If $A$ is commutative, then $M / [A, M] \cong M \otimes_{A \otimes_k A} A$, where $A$ is an $A \otimes_k A$-module via multiplication.
\end{cor}

\subsection{The bar complex}

We'll take $A$ to be commutative from now on, but note that we can make most of this work generally.
For notational simplicity, we'll write $A^e = A \otimes_k A$.

Thinking of $\HHrm_*(A/k, -)$ as the derived functor of $\Delta^*: \QCoh(\Spec A \times_k \Spec A) \to \QCoh(\Spec A)$, we see that
\[
	\HHrm_*(A/k, M) \cong \Tor_*^{A^e}(M, A).
\]
By the symmetry of $\Tor$, we see that it suffices to take a projective resolution of $A$.
This can be accomplished as follows.

\begin{dfn}
	The \emph{bar complex} (or \emph{standard complex}) is the complex of $A^e$-modules
	\[
		\begin{tikzcd}
			\dots \rar["d_2"] & A \otimes_k A \otimes_k A \rar["d_1"] \rar & A \otimes_k A \rar["m"] & A \rar & 0
		\end{tikzcd}
	\]
	where
	\[
		d_n(a_0 \otimes \dots a_{n+1}) = \sum_{i=0}^n (-1)^i a_0 \otimes \dots \otimes a_i a_{i+1} \otimes \dots \otimes a_{n+1} + (-1)^{n+1} a_{n+1} a_0 \otimes \dots \otimes a_n.
	\]
	Here the left (resp.\ right) factor of $A \otimes_k A$ acts on the leftmost (resp.\ rightmost) factor of $A \otimes_k \dots \otimes_k A$.
\end{dfn}

\begin{prop}
	The bar complex is exact.
	Furthermore, as left $A^e$-modules, we have $A^{\otimes n+2} \xrightarrow{\sim} A^e \otimes A^{\otimes n}$ via 
	\[
		a_0 \otimes \dots \otimes a_{n+1} \mapsto (a_0 \otimes a_{n+1}) \otimes a_1 \otimes \dots \otimes a_n.
	\]
	In particular, if $A$ is free as a $k$-module, than $A^{\otimes n+2}$ is free as an $A$-module.
\end{prop}

\begin{thm}
	If $A$ is a free $k$-module, then $\HHrm_*(A/k, M)$ is equivalent to the homology of the complex
	\[
		\begin{tikzcd}
			\dots \rar["d_3"] & M \otimes_k A \otimes_k A \rar["d_2"] \rar & M \otimes_k A \rar["d_1"] & M \rar & 0
		\end{tikzcd}
	\]
	where
	\[
		d_n(m \otimes \dots a_n) = m a_1 \otimes a_2 \otimes \dots \otimes a_n + \sum_{i=1}^{n-1} (-1)^i m \otimes a_1 \otimes \dots \otimes a_i a_{i+1} \otimes \dots \otimes a_n + (-1)^n a_n m \otimes a_1 \otimes \dots \otimes a_n.
	\]
\end{thm}

\begin{proof}
	If $A$ is free, use the standard resolution and $M \otimes_{A^e} (A^{\otimes n+2}) \cong M \otimes_k A^{\otimes n}$.
\end{proof}

\subsection{HKR and computations}

\begin{thm}
	If 
	\[
		\SES{M_1}{M_2}{M_3}
	\]
	is a short exact sequence of $A^e$-modules, we get a LES
	\[
		\begin{tikzcd}
			\dots \rar & \HHrm_1(A/k, M_3) \rar & \HHrm_0(A/k, M_1) \rar & \HHrm_0(A/k, M_2) \rar & \HHrm_0(A/k, M_3) \rar & 0.
		\end{tikzcd}
	\]
	where $\HHrm_0(A/k, M) = M / [A, M] = M \otimes_{A^e} A$.
	Furthermore, if $M$ is a free $A^e$-module, then $\HHrm_i(A/k, M) = 0$ for all $i > 0$.
\end{thm}

\begin{ex}
	Consider the SES
	\[
		\SES{I}{A^e}{A}
	\]
	where $I = \ker(M)$.
	Because $A^e$ is a free module over itself, we get an LES containing the terms
	\[
		\begin{tikzcd}
			0 \rar & \HHrm_1(A / k, A / k) \rar & I \otimes_{A^e} A \rar["\phi"] & A.
		\end{tikzcd}
	\]
	As 
	\[
		\phi\big((a \otimes b) \otimes \alpha\big) = a \alpha b = a b \alpha = 0
	\]
	(because $a \otimes b \in I$), we see
	\[
		\HHrm_1(A / k, A / k) \cong I \otimes_{A^e} A \cong I \otimes_{A^e} A^e / I \cong I / I^2 \cong \Omega^1_{A/k}.
	\]
	Thus $\HHrm_1(A / k, A / k) \cong \Omega^1_{A/k}$, regardless of $A$.
\end{ex}

\begin{thm}[Hochschild-Kostant-Rosenberg]
	There exists a cup product on $\HHrm_*(A / k, A / k)$ and a map $\wedge^* \Omega^1_{A/k} \to \HHrm_*(A / k, A / k)$.
	If $k$ is a field and $A / k$ is smooth, then this map is an isomorphism.
\end{thm}

\section{10/29 (Catherine Cannizzo) -- Homological Mirror Symmetry for Theta Divisors}

I didn't take notes for this talk because the speaker already had well-prepared slides.

\section{11/5 (Martin Olsson) -- Ample Vector Bundles and Projective Geometry of Stacks}

This is based on joint work with Dan Bragg and Rachel Webb.
For simplicity, we work over a field of characteristic zero.

\subsection{Warmup}

Let $g \geq 2$, and consider the Deligne-Mumford moduli space of curves $\ol{\Mc}_g$.
This has a coarse moduli space $\ol{M}_g$.

Recall that $\ol{M}_g$ is a projective scheme.
More precisely, for all $k \geq 1$, we get a line bundle $\Lambda_k$ on $\ol{\Mc}_g$, with sections over $R$-points (corresponding to curves $C$ over $R$) given by $\det H^0(C, \omega_{C/R}^{\otimes k})$.
For $k \gg 0$, this descends to an ample line bundle $\Lc_k$ on $\ol{M}_g$.
Thus we may write 
\[
	\ol{M}_g = \Proj \oplus_{m \geq 0} H^0(\ol{\Mc}_g, \Lambda_k^{\otimes m}) = \Proj \oplus_{m \geq 0} H^0(\ol{M}_g, \Lc_k^{\otimes m}).
\]

\begin{qn}
	How can we produce an analogous description of $\ol{\Mc}_g$?
\end{qn}

\subsection{An example: $\ol{\Mc}_{1,1}$}

Consider $\ol{\Mc}_{1,1}$; this is a ``stacky curve.''\footnote{For more on stacky curves, see the AMS Memoirs book by Voight and Zureick-Brown.}
We define a line bundle $\Lambda$ with sections over an $R$-point $(E, e)$ given by $H^0(E, \omega_{E/R}) = e^* \omega_{E/R}$.
The coarse moduli space here is $\PP^1$ (with the coarse moduli map given by the $j$-invariant).
Mumford showed that $\Lambda^{\otimes 12}$ descends to $\Oc_{\PP^1}(1)$.
	
We'd like to compute the graded ring $\Gamma_*(\ol{\Mc}_{1,1}, \Lambda)$.
Note that $\ol{\Mc}_{1,1}$ has two special points: $i_0: B\mu_6 \to \ol{\Mc}_{1,1}$ and $i_{1728}: B\mu_4 \to \ol{\Mc_{1,1}}$.
Viewing these as divisors $\Dc_0$ and $\Dc_{1728}$, we can show $\Lambda = \Oc_{\ol{\Mc}_{1,1}}(\Dc_{1728} - \Dc_0)$.
From this, we obtain
\[
	H^0(\ol{\Mc}_{1,1}, \Lambda^{\otimes m}) = \begin{cases}
		0 & m \textrm{ odd} \\
		H^0\left(\PP^1, \Oc_{\PP^1}\left( \left\lfloor \frac{m}{4} \right\rfloor - \left\lceil \frac{m}{6} \right\rceil \right)\right) & m \textrm{ even.}
	\end{cases}
\]
This can be used to show that $\Gamma_*(\ol{\Mc}_{1,1}, \Lambda) = k[x, y]$, where $\deg x = 4$ and $\deg y = 6$.

Observe that $\ol{\Mc}_{1,1} \simeq [(\Spec k[x, y] \setminus 0) / \GG_m]$.
This looks like a reasonable ``projective embedding of stacks.''

\subsection{Desiderata}

In general, we should ask:

\begin{qn}
	What plays the role of projective spaces for stacks?
\end{qn}

\begin{qn}
	What plays the role of ample line bundles for stacks?
\end{qn}

The \emph{weighted projective spaces} $\Pc(a_0, \dots, a_n) = [(\AA^{n+1} \setminus 0) /_{\ul{a}} \GG_m]$ should be examples of ``stacky projective spaces.''
However, they shouldn't give all of the examples, as that would only allow cyclic stabilizers.
Similarly, restricting to line bundles only gives embeddings into stacks with cyclic stabilizers.
Abramovich and Hassett study a version of moduli theory for stacks with cyclic stabilizers, but we really want to allow nonabelian examples.

\begin{ex}
	Let's think about the case where $\Xfr = BG$ for a finite group $G$.
	Line bundles on $\Xfr$ correspond to multiplicative characters of $G$.
	Vector bundles on $G$ correspondto representations of $G$ -- knowing these tells us a lot more about $G$!
\end{ex}

\subsection{Generalizing projective spaces}

Let $V$ be a representation of $\GL_r$, and let $\AA_V = \Spec \Sym V$.
Define a subvariety $\AA_V^{s,\det} \subset \AA_V$ by the condition $x \in \AA^{s,\det}_V$ if and only if there exists $f \in \Sym V$ such that 
\begin{enumerate}
	\item $g \cdot f = (\det g)^m f$ for all $g \in \GL_r$,
	\item $f(x) \neq 0$, and
	\item $x$ has closed orbit in $D(f)$.
\end{enumerate}
This should be thought of as similar to a GIT ``stability condition.''

We define $\Qc\Pc(V) = [\AA_v^{s,\det} / \GL_r]$.
By GIT, it follows that $\Qc\Pc(V)$ is a DM stack with finite diagonal.
The coarse moduli space of $\Qc\Pc(V)$ is $\Proj \oplus_{m \geq 0} A_m$, where $A_m$ consists of the $f$ above.

\subsection{Generalizing ample line bundles}

The generalization of ample line bundles used here is inspired by Robin Hartshorne's ``ample vector bundles.''

Let $\Xfr$ be a finite type DM stack with affine diagonal over an algebraically closed field\footnote{We're only assuming this to clean up some of the statements we'll make.} $k$, and let $\pi: \Xfr \to X$ be the coarse moduli space.
For a point $x: \Spec k \to \Xfr$, we write $G_x$ for the corresponding stabilizer group.
In particular, if $\Fc \in \Coh(\Xfr)$, then the fiber $\Fc(x) = x^* \Fc$ is naturally an object of $\Rep(G_x)$.

\begin{dfn}
	Let $\Ec$ be a vector bundle on $\Xfr$.
	We say that $\Ec$ is:
	\begin{enumerate}
		\item \emph{faithful} if, for all $x \in \Xfr(k)$, the representation $\Ec(x)$ of $G_x$ is faithful.
		\item \emph{H-ample} if $\Ec$ is faithful and, for all $\Fc \in \Coh(\Xfr)$, there exists $n_0$ such that $\pi_*(\Fc \otimes \Sym^n \Ec)$ is generated by global sections for all $n \geq n_0$.
		\item \emph{$\det$-ample} if $\Ec$ is faithful and there exists $N > 0$ such that $\det(\Ec)^{\otimes N}$ descends to an ample line bundle on $X$.
	\end{enumerate}
\end{dfn}

\begin{thm}
	The following are equivalent:
	\begin{enumerate}
		\item $X$ is quasiprojective and $\Xfr = [U / \GL_r]$ for some scheme $U$ with action of $\GL_r$ (for some $r \geq 1$)
		\item $\Xfr$ admits a vector bundle that is both H-ample and $\det$-ample.
		\item $\Xfr$ admits a $\det$-ample vector bundle.
		\item $\Xfr$ admits an immersion into $\Qc\Pc(V)$ for some $V$.
	\end{enumerate}
\end{thm}

Given a $\det$-ample vector bundle, how do we obtain a map $\Xfr \to \Qc\Pc(V)$?
Let's forget about stability conditions for the moment and just consider maps from $\Xfr$ to $[\AA_V / \GL_r]$.
These correspond to $\GL_r$-torsors $P \to \Xfr$ together with $\GL_r$-equivariant maps $P \to \AA_V$.
If $P$ corresponds to the rank $r$ vector bundle $\Ec$, then $P$ embeds as an open substack of $\Spec \Sym \Ec^{\oplus r}$.
Requiring that $\Ec$ is faithful implies that $P$ is a \emph{scheme} open in $\Spec_X(\pi_* \Sym \Ec^{\oplus r})$.
This can be used to understand $H^0(P, \Oc_P)$, and the map $P \to \AA_V$ can be constructed as a map $V \to H^0(P, \Oc_P)$.



\end{document}
